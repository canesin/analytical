\documentclass[a4paper,10pt]{article}
\usepackage[utf8x]{inputenc}
\usepackage{amsmath}
\usepackage{geometry}
\geometry{ top=3cm, bottom=2.5cm, left=2.5cm, right=2.5cm}  
%\geometry{papersize={216mm,330mm}, top=3cm, bottom=2.5cm, left=4cm,  right=2cm}  

\newcommand{\D}{\partial}

%opening
\title{Manufactured Solution for Heat Equation using Maple\footnote{Work based on \cite{Kirk2009}.}}
\author{Kemelli C. Estacio-Hiroms}

\begin{document}

% section changes from 
\section math_model Mathematical Model
The transient temperature distribution \f$T({\bf x},t)\f$ in a conducting medium is given by:
\f[
\rho c_p \frac{\D T}{\D t} - \nabla \cdot (k \nabla T) = 0,
\f]
where \f$\rho\f$, \f$c_p\f$ and \f$k\f$ are the material density, specific heat, and thermal conductivity, respectively,
\f$t\f$ denotes time and \f${\bf x}=(x,y,z)\f$.

In this paper, temperature distribution is considered in both steady and unsteady state, with constant and variable material
properties, and in one, two or three-dimensional space, as described in the following sections.

\section{Manufactured Solution}

The general form of the manufactured temperature distribution \f$T = T ({\bf x}, t)\f$ is chosen as a
function of cosines:
\f[
  T ({\bf x}, t) = \cos(A_x x + A_t t) \cdot \cos(B_y y + B_t t) \cdot \cos(C_z z + C_t t) \cdot \cos(D_t t) .
\f]

This solution satisfies the requirements of a suitable manufactured solution: it is smooth,
infinitely differentiable and realizable. Also, it is non-trivial, i.e., it does not vanish even in case of
some of the unknowns are equal to zero.


In sequence, a hierarchy of solutions which test various features of the governing equation is created. This includes
one, two and three-dimensional temperature distribution, steady and transient state, and constant and variable
material properties. For instance, a steady distribution is recovered when
\f$A_t = B_t = C_t = D_t = 0\f$; a one-dimensional distribution results when \f$B = C = 0\f$, while a two-dimensional distribution
results for \f$C = 0\f$. Moreover, while density \f$\rho\f$ is considered constant, thermal capacity \f$c_p\f$ and conductivity \f$k\f$
can be modeled either as constant or variable. In the later case, they are approximated as a polynomial in \f$T\f$.

Although the most general case would be represented by the \f$3D\f$ unsteady distribution with both \f$k\f$ and \f$c_p\f$ variable,
this paper is divided in two main sections: steady and transient distribution. Each one of these sections is subdivided
into other two sections, in case of constant and variable material properties, and source terms obtained for \f$1D\f$, \f$2D\f$ and \f$3D\f$ cases are presented.


\section{Steady Conduction}
For the steady case, our governing equation is simplified to the form:
\f[
 - \nabla \cdot (k \nabla T) = 0.
\f]

Vy considering \f$A_t = B_t = C_t = D_t = 0\f$, it yields:
\f[
  T ({\bf x}) = \cos(A_x x ) \cdot \cos(B_y y ) \cdot \cos(C_z z ).
\f]

In sequence, both approximations for \f$k\f$ (constant or variable) are used in order to obtain source term~ \f$Q\f$ for the steady temperature
distribution from manufactured solution.

\end{document}
