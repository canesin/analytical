\documentclass[10pt]{article}
\usepackage[utf8x]{inputenc}
\usepackage{amsmath}
\usepackage{geometry}
%\usepackage[mathcal]{euscript}
\geometry{ top=2.5cm, bottom=2cm, left=2cm, right=2cm}
\usepackage[authoryear]{natbib}
\usepackage{pdflscape}
%\geometry{papersize={216mm,330mm}, top=3cm, bottom=2.5cm, left=4cm,  right=2cm}

\newcommand{\D}{\partial}
\newcommand{\Diff}[2] {\dfrac{\partial( #1)}{\partial #2}}
\newcommand{\bv}[1]{\ensuremath{\mbox{\boldmath$ #1 $}}}
\newcommand{\gv}[1]{\ensuremath{\mbox{\boldmath$ #1 $}}}% for vectors of Greek letters
\newcommand{\grad}[1]{\gv{\nabla} #1}
\newcommand{\Rho}{\,\mathtt{Rho}}
\newcommand{\PP}{\,\mathtt{P}}
\newcommand{\U}{\,\mathtt{U}}
\newcommand{\V}{\,\mathtt{V}}
\newcommand{\W}{\,\mathtt{W}}
\newcommand{\Lo}{\,\mathcal{L}}
%opening
\title{Manufactured Solution for the Compressible Transient Euler Equations using Maple}
\author{Kemelli C. Estacio-Hiroms}

\begin{document}

\maketitle

\begin{abstract}
The Method of Manufactured Solutions is a valuable approach for code verification, providing means to address how accurately the numerical method solves the equations of interest.
This document presents the source terms generated by the application of the Method of Manufactured Solutions (MMS) on the 1D, 2D, 3D transient Euler equations using the analytical manufactured solutions for density, velocity and pressure presented by \citet{Roy2002}  adapted to handle transient phenomena.
\end{abstract}





\section{Mathematical Model}

The conservation of mass, momentum, and total energy for a compressible transient inviscid fluid may be written as:
\begin{equation}
 \label{eq:euler_01}
\Diff{\rho}{t} +\nabla \cdot \left(\rho \bv{u}\right) = 0,
\end{equation}

\begin{equation}
 \label{eq:euler_02}
\Diff{\rho \bv{u}}{t} +\nabla\cdot\left(\rho\bv{u}\bv{u}\right) = -\nabla p,
\end{equation}

\begin{equation}
 \label{eq:euler_03}
%\nabla \cdot (\rho\bv{u}e_t)+  \nabla\cdot(p  \bv{u})=0
\Diff{\rho e_t}{t} +\nabla\cdot\left(\rho \bv{u} H\right) = 0.
\end{equation}
%

Equations (\ref{eq:euler_01})--(\ref{eq:euler_03}) are known as Euler equations; $\rho$ is the density, $\bv{u}=(u,v,w)$ is the velocity in $x$, $y$ or $z$-direction, respectively,    and $p$ is the pressure. The total enthalpy, $H$, may be expressed in terms of the total energy per unit mass $e_t$, density, and pressure:
$$H = e_t + \dfrac{p}{\rho}.$$

For a calorically perfect gas, the Euler equations are closed with two auxiliary relations for energy:
\begin{equation}
 \label{eq:euler_04}
e_t= e+\dfrac{\bv{u}\cdot \bv{u}}{2},\quad\mbox{and}\quad e=\dfrac{1}{\gamma -1}RT ,
\end{equation}
and with the ideal gas equation of state:
\begin{equation}
 \label{eq:euler_05}
p=\rho RT.
\end{equation}

\section{Manufactured Solution}

The Method of Manufactured Solutions (MMS) applied to Euler equations consists in modifying Equations~(\ref{eq:euler_01})~--~(\ref{eq:euler_03}) by adding a source term to the right-hand side of each equation, so the modified set of equations conveniently has the analytical solution chosen \textit{a priori}.

\citet{Roy2002} introduce the general form of the primitive manufactured solution variables to be  a function of sines and cosines in $x$, $y$ and $z$ only. In this work, \citet{Roy2002}'s manufactured solutions are modified in order to address temporal accuracy as well:
\begin{equation}
 \label{eq:manufactured01}
  \phi (x,y,z,t) = \phi_0+ \phi_x\, f_s \left(\frac{a_{\phi x} \pi x}{L} \right) +  \phi_y \,f_s\left(\frac{a_{\phi y} \pi y}{L}\right) + \phi_z \,f_s\left(\frac{a_{\phi z} \pi z}{L}\right)+ \phi_t \,f_s\left(\frac{a_{\phi t} \pi t}{L}\right),
\end{equation}
where $\phi=\rho,u,v,w$ or $p$, and $f_s(\cdot)$ functions denote either sine or cosine function. Note that in this case, $\phi_x$, $\phi_y$, $\phi_z$  and $\phi_t$ are constants  and the subscripts do not denote differentiation.


Although \citet{Roy2002} provide the constants used in the manufactured solutions for the 2D supersonic and subsonic cases for Euler and Navier-Stokes equations, only the source term for the 2D mass conservation equation~(\ref{eq:euler_01}) is presented.


Source terms  for mass conservation ($Q_\rho$), momentum ($Q_u$, $Q_v$ and $Q_w$)  and total energy ($Q_{e_t}$) equations are obtained by symbolic manipulations of compressible transient Euler equations above using Maple~13~\citep{Maple} and are presented in the following sections for  one, two and three-dimensional cases.




\section{1D Transient Euler equations}

The manufactured analytical solutions (\ref{eq:manufactured01}) for each one of the variables in the 1D case of Euler equations are:
\begin{equation}
\begin{split}
\label{eq:manufactured_1d}
\rho\left(x,t\right) &=  \rho_{0}+ \rho_{x} \sin\left(\frac{a_{ \rho x} \pi x}{L}\right)+ \rho_t \sin\left(\dfrac{a_{\rho t} \pi t}{L}\right),\\
u\left(x,t\right) &= u_{0}+u_{x} \sin\left(\frac{a_{u x} \pi x}{L}\right) + u_t \cos\left(\dfrac{a_{u t} \pi t}{L}\right),\\
p\left(x,t\right) &= p_{0}+p_{x} \cos\left(\frac{a_{p x} \pi x}{L}\right)+ p_t \cos\left(\dfrac{a_{p t} \pi t}{L}\right).\\
\end{split}
\end{equation}


The MMS applied to Euler equations consists in modifying the 1D Euler equations~(\ref{eq:euler_01}) -- (\ref{eq:euler_03}) by adding a source term to the right-hand side of each equation:
\begin{equation}
 \label{eq:euler_mod_2d}
\begin{split}
\Diff{\rho}{t} +\Diff{\rho u}{x} &= Q_\rho,\\
\Diff{\rho u}{t} + \Diff{\rho u^2 }{x}+ \Diff{p}{x} &= Q_u,\\
\Diff{\rho e_t}{t} +\Diff{\rho ue_t}{x}+ \Diff{pu}{x} &= Q_{e_t},
\end{split}
\end{equation}
%
so the modified set of equations (\ref{eq:euler_mod_2d}) conveniently has the analytical solution given in Equation (\ref{eq:manufactured_1d}).
%

Source terms $Q_\rho$, $Q_u$ and $Q_{e_t}$ are obtained by symbolic manipulations of equations above using Maple and are presented in the following sections. The following auxiliary variables have been included in order to improve readability and computational efficiency:
\begin{equation*}
 \begin{split}
\label{eq:aux_1d}
\Rho_1 &= \rho_{0}+ \rho_{x} \sin\left(\frac{a_{ \rho x} \pi x}{L}\right) + \rho_t \sin\left(\dfrac{a_{\rho t} \pi t}{L}\right),\\
\U_1 &= u_{0}+u_{x} \sin\left(\frac{a_{u x} \pi x}{L}\right) + u_t \cos\left(\dfrac{a_{u t} \pi t}{L}\right),\\
\PP_1 &= p_{0}+p_{x} \cos\left(\frac{a_{p x} \pi x}{L}\right)+ p_t \cos\left(\dfrac{a_{p t} \pi t}{L}\right),\\
\end{split}
\end{equation*}
where the subscripts in $\Rho_1$, $\PP_1$ and $\U_1$ refer to the 1D case.

\subsection{1D Mass Conservation}

The mass conservation equation written as an operator is:
\begin{equation*}
 \Lo=\Diff{\rho}{t} + \Diff{\rho u}{x}.
\end{equation*}

Analytically differentiating Equation (\ref{eq:manufactured_1d}) for $\rho$ and $u$  using operator $\Lo$ defined above gives  the source term $Q_{\rho}$:
\begin{equation}
 \begin{split}
Q_\rho &= \dfrac{a_{\rho x} \pi \rho_x \U_1}{L}\cos\left(\dfrac{a_{\rho x} \pi x}{L}\right) +\\
&+\dfrac{a_{ux} \pi u_x \Rho_1}{L}\cos\left(\dfrac{a_{ux} \pi x}{L}\right)+\\
&+  \dfrac{ a_{\rho t} \pi \rho_t}{L}\cos\left(\dfrac{a_{\rho t} \pi t}{L}\right) .
 \end{split}
\end{equation}


\subsection{1D Momentum}

For the generation of the analytical source term $Q_u$ for the $x$-momentum equation, Equation  (\ref{eq:euler_02}) is written as an  operator $\Lo$:
\begin{equation*}
 \Lo=\Diff{\rho u}{t} + \Diff{\rho u^2 }{x}+ \Diff{p}{x},
\end{equation*}
which, when operated in Equation (\ref{eq:manufactured_1d}), provides source term $Q_{u}$:

\begin{equation}
 \begin{split}
Q_u&=\dfrac{a_{\rho x} \pi \rho_x \U_1^2}{L}\cos\left(\dfrac{a_{\rho x} \pi x}{L}\right) +\\
&-\dfrac{ a_{px} \pi p_x}{L}\sin\left(\dfrac{a_{px} \pi x}{L}\right) +\\
&+ \dfrac{2a_{ux} \pi u_x \Rho_1 \U_1}{L}\cos\left(\dfrac{a_{ux} \pi x}{L}\right)+\\
&+ \dfrac{a_{\rho t} \pi \rho_t \U_1}{L}\cos\left(\dfrac{a_{\rho t} \pi t}{L}\right) +\\
&-\dfrac{a_{ut} \pi u_t \Rho_1}{L}\sin\left(\dfrac{a_{ut} \pi t}{L}\right).
 \end{split}
\end{equation}



\subsection{1D Total Energy}


The last component of Euler equations is written as an operator:
\begin{equation*}
 \Lo=\Diff{\rho e_t}{t} + \Diff{\rho ue_t }{x}+\Diff{pu}{x} .
\end{equation*}


Source term $Q_e$ is obtained by operating $\Lo$ on Equation  (\ref{eq:manufactured_1d}) together with the use of the  auxiliary relations~(\ref{eq:euler_04})--(\ref{eq:euler_05}) for energy:
  \begin{equation}\label{eq:source_e}
 \begin{split}
Q_e &= \dfrac{a_{\rho x} \pi \rho_x \U_1^3 }{2L}\cos\left(\dfrac{a_{\rho x} \pi x}{L}\right)+\\
&- \dfrac{\gamma}{\gamma-1}\dfrac{a_{px} \pi p_x \U_1}{L}\sin\left(\dfrac{a_{px} \pi x}{L}\right) +\\
&+ \dfrac{\gamma}{\gamma-1}\dfrac{a_{ux} \pi u_x \PP_1}{L}\cos\left(\dfrac{a_{ux} \pi x}{L}\right)  +\\
&+ \dfrac{ 3 a_{ux} \pi u_x \Rho_1 \U_1^2 }{2L}\cos\left(\dfrac{a_{ux} \pi x}{L}\right)+\\
&+   \dfrac{a_{\rho t} \pi \rho_t \U_1^2}{2L}\cos\left(\dfrac{a_{\rho t} \pi t}{L}\right)+\\
&- \dfrac{1}{\gamma-1}\dfrac{a_{pt} \pi p_t}{L}\sin\left(\dfrac{a_{pt} \pi t}{L}\right)+\\
&-\dfrac{ a_{ut} \pi u_t \Rho_1 \U_1}{L}\sin\left(\dfrac{a_{ut} \pi t}{L}\right).
 \end{split}
\end{equation}

 \section{2D Transient Euler equations}
The manufactured analytical solutions (\ref{eq:manufactured01}) for each one of the variables in the two-dimensional Euler equations are:
\begin{equation}
\begin{split}
\label{eq:manufactured_2d}
\rho\left(x,y,t\right) &=  \rho_{0}+ \rho_{x} \sin\left(\frac{a_{ \rho x} \pi x}{L}\right)+ \rho_{y} \cos\left(\frac{a_{ \rho y} \pi y}{L}\right)+ \rho_t \sin\left(\dfrac{a_{\rho t} \pi t}{L}\right),\\
u\left(x,y,t\right) &= u_{0}+u_{x} \sin\left(\frac{a_{u x} \pi x}{L}\right)+u_{y} \cos\left(\frac{a_{u y} \pi y}{L}\right) + u_t \cos\left(\dfrac{a_{u t} \pi t}{L}\right),\\
v\left(x,y,t\right) &= v_{0}+v_{x} \cos\left(\frac{a_{v x} \pi x}{L}\right)+v_{y} \sin\left(\frac{a_{v y} \pi y}{L}\right)+ v_t \sin\left(\dfrac{a_{v t} \pi t}{L}\right),\\
p\left(x,y,t\right) &= p_{0}+p_{x} \cos\left(\frac{a_{p x} \pi x}{L}\right)+p_{y} \sin\left(\frac{a_{p y} \pi y}{L}\right)+ p_t \cos\left(\dfrac{a_{p t} \pi t}{L}\right).\\
\end{split}
\end{equation}


Analogously to the 1D case, MMS applied to the 2D transient Euler consists in modifying Equations~(\ref{eq:euler_01}) -- (\ref{eq:euler_03}) by adding a source term to the right-hand side of each equation:
\begin{equation}
\begin{split}
\label{eq:euler2d_mod}
&\Diff{\rho}{t} + \Diff{\rho u}{x}+\Diff{\rho v}{y} = Q_\rho\\
&\Diff{\rho u}{t} +  \Diff{\rho u^2 }{x}+\Diff{\rho uv}{y} +\Diff{p}{x}= Q_u\\
&\Diff{\rho v}{t} +  \Diff{\rho vu}{x}+\Diff{\rho v^2}{y} +\Diff{p}{y}= Q_v\\
& \Diff{\rho e_t}{t} +\Diff{\rho ue_t }{x}+\Diff{\rho ve_t}{y}+\Diff{pu}{x} +\Diff{pv}{y}= Q_{e_t}
\end{split}
\end{equation}
so the modified set of Equations (\ref{eq:euler2d_mod}) has Equation (\ref{eq:manufactured_2d}) as analytical solution.


Source terms $Q_\rho$, $Q_u$, $Q_v$ and $Q_{e_t}$ are presented in the subsequent sections with the use of the following auxiliary variables in order to improve readability and computational efficiency:
\begin{equation*}
 \begin{split}
\label{eq:aux_2d}
\Rho_2 &= \rho_{0}+ \rho_{x} \sin\left(\frac{a_{ \rho x} \pi x}{L}\right)+ \rho_{y} \cos\left(\frac{a_{ \rho y} \pi y}{L}\right)+ \rho_t \sin\left(\dfrac{a_{\rho t} \pi t}{L}\right),\\
\U_2 &= u_{0}+u_{x} \sin\left(\frac{a_{u x} \pi x}{L}\right)+u_{y} \cos\left(\frac{a_{u y} \pi y}{L}\right) + u_t \cos\left(\dfrac{a_{u t} \pi t}{L}\right),\\
\V_2 &= v_{0}+v_{x} \cos\left(\frac{a_{v x} \pi x}{L}\right)+v_{y} \sin\left(\frac{a_{v y} \pi y}{L}\right)+ v_t \sin\left(\dfrac{a_{v t} \pi t}{L}\right),\\
\PP_2 &= p_{0}+p_{x} \cos\left(\frac{a_{p x} \pi x}{L}\right)+p_{y} \sin\left(\frac{a_{p y} \pi y}{L}\right)+ p_t \cos\left(\dfrac{a_{p t} \pi t}{L}\right),\\
\end{split}
\end{equation*}
%
where the subscripts in $\Rho_2$, $\PP_2$, $\U_2$ and $\V_2$ refer to the 2D case.

\subsection{2D Mass Conservation}

The 2D mass conservation equation written as an operator is:
\begin{equation*}
 \Lo=\Diff{\rho e_t}{t} +  \Diff{\rho}{t} +\Diff{\rho u}{x}+\Diff{\rho v}{y}.
\end{equation*}

Analytically differentiating Equation (\ref{eq:manufactured_2d}) for $\rho$, $u$ and $v$ using operator $\Lo$ defined above gives  the source term~$Q_{\rho}$:
\begin{equation}
 \begin{split}
Q_\rho &= \dfrac{a_{\rho x} \pi \rho_x \U_2 }{L}\cos\left(\dfrac{a_{\rho x} \pi x}{L}\right)+\\
&-\dfrac{a_{\rho y} \pi \rho_y \V_2 }{L}\sin\left(\dfrac{a_{\rho y} \pi y}{L}\right)+\\
&+\dfrac{\pi \Rho_2}{L}\left[a_{ux} u_x \cos\left(\dfrac{a_{ux} \pi x}{L}\right)+a_{vy} v_y  \cos\left(\dfrac{a_{vy} \pi y}{L}\right)\right]+\\
&+  \dfrac{a_{\rho t} \pi \rho_t }{L}\cos\left(\dfrac{a_{\rho t} \pi t}{L}\right).
 \end{split}
\end{equation}

\subsection{2D Momentum}

For the generation of the analytical source term $Q_u$ for the $x$-momentum equation, the first component of Equation~(\ref{eq:euler_02}) is written as an  operator $\Lo$:
\begin{equation*}
 \Lo= \Diff{\rho u}{t} + \Diff{\rho u^2}{x}+\Diff{\rho uv}{y} + \Diff{p}{x},
\end{equation*}
which, when operated in Equation (\ref{eq:manufactured_2d}), provides source term $Q_{u}$:

\begin{equation}
 \begin{split}
Q_u &=  \dfrac{a_{\rho x} \pi \rho_x \U_2^2}{L}\cos\left(\dfrac{a_{\rho x} \pi x}{L}\right)+\\
&- \dfrac{a_{\rho y} \pi \rho_y \U_2 \V_2}{L}\sin\left(\dfrac{a_{\rho y} \pi y}{L}\right)+\\
&-\dfrac{a_{px} \pi p_x }{L}\sin\left(\dfrac{a_{px} \pi x}{L}\right)+\\
&+\dfrac{\pi \Rho_2 \U_2}{L}\left[2 a_{ux} u_x \cos\left(\dfrac{a_{ux} \pi x}{L}\right)+a_{vy} v_y  \cos\left(\dfrac{a_{vy} \pi y}{L}\right)\right]+\\
&-\dfrac{a_{uy} \pi u_y \Rho_2 \V_2 }{L}\sin\left(\dfrac{a_{uy} \pi y}{L}\right)+\\
&+  \dfrac{a_{\rho t} \pi \rho_t \U_2 }{L}\cos\left(\dfrac{a_{\rho t} \pi t}{L}\right)+\\
&-\dfrac{a_{ut} \pi u_t \Rho_2 }{L}\sin\left(\dfrac{a_{ut} \pi t}{L}\right).
 \end{split}
\end{equation}


Analogously, for the generation of the analytical source term $Q_v$ for the $y$-momentum equation, the second component of Equation  (\ref{eq:euler_02})  is written as an  operator $\Lo$:
\begin{equation*}
   \Lo =\Diff{\rho v}{t} + \Diff{\rho u v}{x} + \Diff{\rho  v^2 }{y}+ \Diff{p}{y},
\end{equation*}
and then applied to Equation  (\ref{eq:manufactured_2d}). It yields:
\begin{equation}
 \begin{split}
Q_v &=\dfrac{ a_{\rho x} \pi \rho_x \U_2 \V_2 }{L}\cos\left(\dfrac{a_{\rho x} \pi x}{L}\right)+\\
&-\dfrac{a_{\rho y} \pi \rho_y \V_2^2 }{L}\sin\left(\dfrac{a_{\rho y} \pi y}{L}\right)+\\
&+\dfrac{a_{py} \pi p_y}{L}\cos\left(\dfrac{a_{py} \pi y}{L}\right) +\\
&-\dfrac{ a_{vx} \pi v_x \Rho_2 \U_2}{L}\sin\left(\dfrac{a_{vx} \pi x}{L}\right)+\\
&+\dfrac{\pi \Rho_2 \V_2}{L}\left[a_{ux} u_x \cos\left(\dfrac{a_{ux} \pi x}{L}\right)+2 a_{vy} v_y  \cos\left(\dfrac{a_{vy} \pi y}{L}\right)\right]+\\
&+  \dfrac{a_{\rho t} \pi \rho_t \V_2 }{L}\cos\left(\dfrac{a_{\rho t} \pi t}{L}\right)+\\
&+\dfrac{a_{vt} \pi v_t \Rho_2 }{L}\cos\left(\dfrac{a_{vt} \pi t}{L}\right).
 \end{split}
\end{equation}



\subsection{2D Total Energy}

The operator for the 2D Euler total energy is:
\begin{equation*}
 \Lo=  \Diff{\rho ue_t }{x}+\Diff{\rho ve_t}{y}+\Diff{pu}{x} +\Diff{pv}{y}.
\end{equation*}


Source term $Q_{e_t}$ is obtained by operating $\Lo$ on Equation  (\ref{eq:manufactured_2d}) together with the use of the  auxiliary relations~(\ref{eq:euler_04})--(\ref{eq:euler_05}) for energy:

\begin{equation}
 \begin{split}\label{eq:source_e_2d}
Q_{e_t} &=\dfrac{ a_{\rho x} \pi \rho_x  \U_2 (\U_2^2+\V_2^2)}{2L}\cos\left(\dfrac{a_{\rho x} \pi x}{L}\right)+\\
&-\dfrac{ a_{\rho y} \pi \rho_y \V_2 (\U_2^2+\V_2^2) }{2L}\sin\left(\dfrac{a_{\rho y} \pi y}{L}\right) +\\
&-\dfrac{\gamma}{\gamma-1}\dfrac{a_{px} \pi p_x  \U_2}{L}\sin\left(\dfrac{a_{px} \pi x}{L}\right) +\\
&+  \dfrac{\gamma}{\gamma-1}\dfrac{a_{py}\pi p_y  \V_2}{L}\cos\left(\dfrac{a_{py} \pi y}{L}\right)+\\
&+ \dfrac{ \pi \Rho_2 \U_2^2}{2L}\left[3 a_{ux} u_x \cos\left(\dfrac{a_{ux} \pi x}{L}\right)+a_{vy} v_y \cos\left(\dfrac{a_{vy} \pi y}{L}\right)\right]+\\
&-\dfrac{\pi \Rho_2 \U_2 \V_2}{L}\left[a_{uy} u_y \sin\left(\dfrac{a_{uy} \pi y}{L}\right)+a_{vx}  v_x\sin\left(\dfrac{a_{vx} \pi x}{L}\right)\right] +\\
&+\dfrac{\pi \Rho_2 \V_2^2}{2L} \left[a_{ux} u_x \cos\left(\dfrac{a_{ux} \pi x}{L}\right)+3 a_{vy} v_y \cos\left(\dfrac{a_{vy} \pi y}{L}\right)\right] +\\
&+ \dfrac{\gamma}{\gamma-1}\dfrac{\pi \PP_2}{L}\left[a_{ux} u_x \cos\left(\dfrac{a_{ux} \pi x}{L}\right)+a_{vy} v_y \cos\left(\dfrac{a_{vy} \pi y}{L}\right)\right]+\\
&+\dfrac{  a_{\rho t} \pi \rho_t (\U_2^2+\V_2^2)}{2L}\cos\left(\dfrac{a_{\rho t} \pi t}{L}\right)+\\
&-\dfrac{1}{\gamma-1}\dfrac{a_{pt} \pi p_t }{L}\sin\left(\dfrac{a_{pt} \pi t}{L}\right) +\\
&-\dfrac{a_{ut} \pi u_t \Rho_2 \U_2 }{L}\sin\left(\dfrac{a_{ut} \pi t}{L}\right)+\\
&+\dfrac{a_{vt} \pi v_t \Rho_2 \V_2 }{L}\cos\left(\dfrac{a_{vt} \pi t}{L}\right).
 \end{split}
\end{equation}



\section{3D Transient Euler equations}

The manufactured analytical solution for for each one of the variables in 3D transient Euler equations are:
\begin{equation}
\begin{split}
\label{eq:manufactured_3d}
\rho\left( x ,y ,z,t\right) &=  \rho_{0}+ \rho_{x} \sin\left(\frac{a_{ \rho  x} \pi x}{L}\right)+ \rho_{y} \cos\left(\frac{a_{ \rho  y} \pi y}{L}\right) + \rho_{z} \sin\left(\frac{a_{ \rho  z} \pi z}{L}\right) + \rho_t \sin\left(\dfrac{a_{\rho t} \pi t}{L}\right),\\
u\left( x ,y ,z,t\right) &= u_{0}+u_{x} \sin\left(\frac{a_{u  x} \pi x}{L}\right)+u_{y} \cos\left(\frac{a_{u  y} \pi y}{L}\right)+u_{z} \cos\left(\frac{a_{u  z} \pi z}{L}\right) + u_t \cos\left(\dfrac{a_{u t} \pi t}{L}\right),\\
v\left( x ,y ,z,t\right) &= v_{0}+v_{x} \cos\left(\frac{a_{v  x} \pi x}{L}\right)+v_{y} \sin\left(\frac{a_{v  y} \pi y}{L}\right)+v_{z} \sin\left(\frac{a_{v  z} \pi z}{L}\right)+ v_t \sin\left(\dfrac{a_{v t} \pi t}{L}\right), \\
w\left( x ,y ,z,t\right) &= w_{0}+w_{x} \sin\left(\frac{a_{w  x} \pi x}{L}\right)+w_{y} \sin\left(\frac{a_{w  y} \pi y}{L}\right)+ w_{z} \cos\left(\frac{a_{w  z} \pi z}{L}\right)+ w_t \cos\left(\dfrac{a_{w t} \pi t}{L}\right) ,\\
p\left( x ,y ,z,t\right) &= p_{0}+p_{x} \cos\left(\frac{a_{p  x} \pi x}{L}\right)+p_{y} \sin\left(\frac{a_{p  y} \pi y}{L}\right)+ p_{z} \cos\left(\frac{a_{p  z} \pi z}{L}\right)+ p_t \cos\left(\dfrac{a_{p t} \pi t}{L}\right).\\
\end{split}
\end{equation}


The MMS applied to 3D Euler equations consists in modifying Equations~(\ref{eq:euler_01})--(\ref{eq:euler_03}) by adding a source term to the right-hand side of each equation:

\begin{equation}
\begin{split}
\label{eq:euler_3d_mod}
&\Diff{\rho}{t} +\Diff{\rho u}{x}+\Diff{\rho v}{y} + \Diff{\rho w}{z} = Q_\rho\\
%
 &\Diff{\rho u}{t} + \Diff{\rho u^2 }{x}+\Diff{\rho uv}{y} +\Diff{\rho uw}{z} +\Diff{p}{x}= Q_u,\\
%
&\Diff{\rho v}{t} + \Diff{\rho vu }{x}+\Diff{\rho v^2}{y} +\Diff{\rho vw}{z}+\Diff{p}{y}= Q_v,\\
%
&\Diff{\rho w}{t} + \Diff{\rho w u }{ x}+\Diff{\rho w v }{ y}+\Diff{\rho w^2 }{ z}+\Diff{p}{z}=Q_w,\\
%
&\Diff{\rho e_t}{t} +\Diff{\rho u e_t}{x}+\Diff{\rho v e_t}{y}+\Diff{\rho w e_t}{z}+\Diff{pu}{x}+\Diff{pv}{y}+\Diff{pw}{z} =Q_{e_t},
\end{split}
\end{equation}
so this modified set of equations has for analytical solution Equation (\ref{eq:manufactured_3d}).

Analogously to the 1D and 2D cases, the source terms $Q_\rho$, $Q_u$, $Q_v$, $Q_w$ and $Q_{e_t}$ are  presented with the use of the following auxiliary variables:
\begin{equation*}
 \begin{split}
\label{eq:aux}
\Rho_3 &= \rho_{0}+ \rho_{x} \sin\left(\frac{a_{ \rho  x} \pi x}{L}\right)+ \rho_{y} \cos\left(\frac{a_{ \rho  y} \pi y}{L}\right) + \rho_{z} \sin\left(\frac{a_{ \rho  z} \pi z}{L}\right)+ \rho_t \sin\left(\dfrac{a_{\rho t} \pi t}{L}\right),\\
\U_3 &=u_{0}+u_{x} \sin\left(\frac{a_{u  x} \pi x}{L}\right)+u_{y} \cos\left(\frac{a_{u  y} \pi y}{L}\right)+u_{z} \cos\left(\frac{a_{u  z} \pi z}{L}\right) + u_t \cos\left(\dfrac{a_{u t} \pi t}{L}\right) ,\\
\V_3 &= v_{0}+v_{x} \cos\left(\frac{a_{v  x} \pi x}{L}\right)+v_{y} \sin\left(\frac{a_{v  y} \pi y}{L}\right)+v_{z} \sin\left(\frac{a_{v  z} \pi z}{L}\right)+ v_t \sin\left(\dfrac{a_{v t} \pi t}{L}\right), \\
\W_3 &= w_{0}+w_{x} \sin\left(\frac{a_{w  x} \pi x}{L}\right)+w_{y} \sin\left(\frac{a_{w  y} \pi y}{L}\right)+ w_{z} \cos\left(\frac{a_{w  z} \pi z}{L}\right) + w_t \cos\left(\dfrac{a_{w t} \pi t}{L}\right),\\
\PP_3 &= p_{0}+p_{x} \cos\left(\frac{a_{p  x} \pi x}{L}\right)+p_{y} \sin\left(\frac{a_{p  y} \pi y}{L}\right)+ p_{z} \cos\left(\frac{a_{p  z} \pi z}{L}\right)+ p_t \cos\left(\dfrac{a_{p t} \pi t}{L}\right).\\
\end{split}
\end{equation*}
where, again, the subscripts in $\Rho_3$, $\PP_3$, $\U_3$, $\V_3$ and $\W_3$ refer to the 3D case.

\subsection{3D Mass Conservation}

The source term $Q_{\rho}$ for the 3D mass conservation equation is:
\begin{equation}
 \begin{split}
Q_\rho &=\dfrac{ a_{\rho x} \pi \rho_x \U_3}{L} \cos\left(\dfrac{a_{\rho x} \pi x}{L}\right)+\\
&-\dfrac{a_{\rho y} \pi \rho_y \V_3 }{L}\sin\left(\dfrac{a_{\rho y} \pi y}{L}\right)+\\
&+\dfrac{a_{\rho z} \pi \rho_z \W_3 }{L}\cos\left(\dfrac{a_{\rho z}\pi z }{L}\right)+\\
&+\dfrac{\pi \Rho_3}{L}\left[a_{ux} u_x \cos\left(\dfrac{a_{ux} \pi x}{L}\right)+a_{vy} v_y \cos\left(\dfrac{a_{vy} \pi y}{L}\right)-a_{wz} w_z \sin\left(\dfrac{a_{wz}\pi z }{L}\right)\right] +\\
&+  \dfrac{a_{\rho t} \pi \rho_t }{L}\cos\left(\dfrac{a_{\rho t} \pi t}{L}\right).
 \end{split}
\end{equation}

\subsection{3D Momentum}

The source terms $Q_{u}$, $Q_{v}$ and $Q_{w}$ for the 3D momentum equations the in $x$, $y$ and $z$ directions are, respectively:
\begin{equation}
 \begin{split}
Q_u &= \dfrac{a_{\rho x} \pi \rho_x \U_3^2 }{L}\cos\left(\dfrac{a_{\rho x} \pi x}{L}\right)+\\
&-\dfrac{a_{\rho y} \pi \rho_y \U_3 \V_3 }{L}\sin\left(\dfrac{a_{\rho y} \pi y}{L}\right)+\\
&+\dfrac{a_{\rho z} \pi \rho_z \U_3 \W_3 }{L}\cos\left(\dfrac{a_{\rho z}\pi z }{L}\right)+\\
&-\dfrac{a_{px} \pi p_x }{L}\sin\left(\dfrac{a_{px} \pi x}{L}\right)+\\
&+\dfrac{\pi \Rho_3 \U_3}{L}\left[2 a_{ux} u_x \cos\left(\dfrac{a_{ux} \pi x}{L}\right)+a_{vy} v_y \cos\left(\dfrac{a_{vy} \pi y}{L}\right)-a_{wz} w_z \sin\left(\dfrac{a_{wz}\pi z }{L}\right)\right]+\\
&-\dfrac{a_{uy} \pi u_y \Rho_3 \V_3}{L} \sin\left(\dfrac{a_{uy} \pi y}{L}\right)+\\
&-\dfrac{a_{uz} \pi u_z \Rho_3 \W_3 }{L}\sin\left(\dfrac{a_{uz}\pi z }{L}\right)+\\
&+  \dfrac{ a_{\rho t} \pi \rho_t \U_3}{L}\cos\left(\dfrac{a_{\rho t} \pi  t}{L}\right)+\\
&-\dfrac{a_{ut} \pi u_t \Rho_3 }{L}\sin\left(\dfrac{a_{ut} \pi  t}{L}\right),
 \end{split}
\end{equation}

\begin{equation}
 \begin{split}
Q_v &= \dfrac{a_{\rho x} \pi \rho_x \U_3 \V_3 }{L}\cos\left(\dfrac{a_{\rho x} \pi x}{L}\right)+\\
&-\dfrac{a_{\rho y} \pi \rho_y \V_3^2 }{L}\sin\left(\dfrac{a_{\rho y} \pi y}{L}\right)+\\
&+\dfrac{a_{\rho z} \pi \rho_z \V_3 \W_3 }{L}\cos\left(\dfrac{a_{\rho z}\pi z }{L}\right)+\\
&+\dfrac{a_{py} \pi p_y }{L}\cos\left(\dfrac{a_{py} \pi y}{L}\right)+\\
&-\dfrac{a_{vx} \pi v_x \Rho_3 \U_3 }{L}\sin\left(\dfrac{a_{vx} \pi x}{L}\right)+\\
&+\dfrac{\pi \Rho_3 \V_3}{L}\left[a_{ux} u_x \cos\left(\dfrac{a_{ux} \pi x}{L}\right)+2 a_{vy} v_y \cos\left(\dfrac{a_{vy} \pi y}{L}\right)-a_{wz} w_z \sin\left(\dfrac{a_{wz}\pi z }{L}\right)\right]+\\
&+\dfrac{a_{vz} \pi v_z \Rho_3 \W_3 }{L}\cos\left(\dfrac{a_{vz}\pi z }{L}\right) +\\
&+  \dfrac{ a_{\rho t} \pi\rho_t \V_3 }{L}\cos\left(\dfrac{a_{\rho t} \pi  t}{L}\right)+\\
&+\dfrac{a_{vt} \pi v_t \Rho_3 }{L}\cos\left(\dfrac{a_{vt} \pi t}{L}\right),
 \end{split}
\end{equation}
and
\begin{equation}
 \begin{split}
Q_w &= \dfrac{a_{\rho x} \pi \rho_x \U_3 \W_3}{L} \cos\left(\dfrac{a_{\rho x} \pi x}{L}\right)+\\
&-\dfrac{a_{\rho y} \pi \rho_y \V_3 \W_3 }{L}\sin\left(\dfrac{a_{\rho y} \pi y}{L}\right)+\\
&+\dfrac{a_{\rho z} \pi \rho_z \W_3^2 }{L}\cos\left(\dfrac{a_{\rho z}\pi z }{L}\right)+\\
&-\dfrac{a_{pz} \pi p_z }{L}\sin\left(\dfrac{a_{pz}\pi z }{L}\right)+\\
&+\dfrac{a_{wx} \pi w_x \Rho_3 \U_3 }{L}\cos\left(\dfrac{a_{wx} \pi x}{L}\right)+\\
&+\dfrac{a_{wy} \pi w_y \Rho_3 \V_3}{L}\cos\left(\dfrac{a_{wy} \pi y}{L}\right) +\\
&+\dfrac{\pi \Rho_3 \W_3}{L}\left[a_{ux} u_x \cos\left(\dfrac{a_{ux} \pi x}{L}\right)+a_{vy} v_y \cos\left(\dfrac{a_{vy} \pi y}{L}\right)-2 a_{wz} w_z \sin\left(\dfrac{a_{wz}\pi z }{L}\right)\right]+\\
&+ \dfrac{a_{\rho t} \pi \rho_t \W_3 }{L}\cos\left(\dfrac{a_{\rho t} \pi  t}{L}\right)+\\
 &-\dfrac{a_{wt} \pi w_t \Rho_3 }{L}\sin\left(\dfrac{a_{wt} \pi  t}{L}\right).
 \end{split}
\end{equation}

\subsection{3D Total Energy}

Finally, the source term $Q_{e_t}$ for the 3D total energy equation is:
\begin{equation}
 \begin{split}\label{eq:euler_3d_e}
Q_{e_t} &= \dfrac{a_{\rho x} \pi \rho_x \U_3 (\U_3^2+\V_3^2+\W_3^2)}{2L}\cos\left(\dfrac{a_{\rho x} \pi x}{L}\right)+\\
&- \dfrac{a_{\rho y} \pi \rho_y \V_3 (\U_3^2+\V_3^2+\W_3^2)}{2L}\sin\left(\dfrac{a_{\rho y} \pi y}{L}\right)+\\
&+  \dfrac{a_{\rho z} \pi \rho_z \W_3 (\U_3^2+\V_3^2+\W_3^2)}{2L}\cos\left(\dfrac{a_{\rho z} \pi z}{L}\right)+\\
&-\dfrac{\gamma}{\gamma-1}\dfrac{a_{px} \pi p_x  \U_3}{L}\sin\left(\dfrac{a_{px} \pi x}{L}\right) +\\
&+\dfrac{\gamma}{\gamma-1}\dfrac{a_{py} \pi p_y \V_3}{L}\cos\left(\dfrac{a_{py} \pi y}{L}\right) +\\
&-\dfrac{\gamma}{\gamma-1}\dfrac{a_{pz}\pi  p_z  \W_3}{L}\sin\left(\dfrac{a_{pz} \pi z}{L}\right) +\\
&+ \dfrac{\gamma}{\gamma-1}\dfrac{\PP_3 \pi}{L}\left[a_{ux} u_x \cos\left(\dfrac{a_{ux} \pi x}{L}\right)+a_{vy} v_y \cos\left(\dfrac{a_{vy} \pi y}{L}\right)-a_{wz} w_z \sin\left(\dfrac{a_{wz} \pi z}{L}\right)\right] +\\
&+\dfrac{\pi \Rho_3 \U_3^2}{2L}\left[3 a_{ux} u_x \cos\left(\dfrac{a_{ux} \pi x}{L}\right)+a_{vy} v_y \cos\left(\dfrac{a_{vy} \pi y}{L}\right)-a_{wz} w_z \sin\left(\dfrac{a_{wz} \pi z}{L}\right)\right]+\\
&-\dfrac{\pi \Rho_3 \U_3 \V_3}{L}\left[a_{uy} u_y \sin\left(\dfrac{a_{uy} \pi y}{L}\right)+a_{vx} v_x \sin\left(\dfrac{a_{vx} \pi x}{L}\right)\right] +\\
&-\dfrac{\pi \Rho_3 \U_3 \W_3}{L}\left[a_{uz} u_z \sin\left(\dfrac{a_{uz} \pi z}{L}\right)-a_{wx} w_x \cos\left(\dfrac{a_{wx} \pi x}{L}\right)\right] +\\
&+\dfrac{\pi \Rho_3 \V_3^2}{2L}\left[a_{ux} u_x \cos\left(\dfrac{a_{ux} \pi x}{L}\right)+3 a_{vy} v_y \cos\left(\dfrac{a_{vy} \pi y}{L}\right)-a_{wz} w_z \sin\left(\dfrac{a_{wz} \pi z}{L}\right)\right] +\\
&+ \dfrac{\pi \Rho_3 \V_3 \W_3}{L}\left[a_{vz} v_z \cos\left(\dfrac{a_{vz} \pi z}{L}\right)+a_{wy} w_y \cos\left(\dfrac{a_{wy} \pi y}{L}\right)\right]+\\
&+\dfrac{\pi \Rho_3 \W_3^2}{2L}\left[a_{ux} u_x \cos\left(\dfrac{a_{ux} \pi x}{L}\right)+a_{vy} v_y \cos\left(\dfrac{a_{vy} \pi y}{L}\right)-3 a_{wz} w_z \sin\left(\dfrac{a_{wz} \pi z}{L}\right)\right] +\\
&+   \dfrac{ a_{\rho t} \pi \rho_t(\U_3^2+\V_3^2+\W_3^2)}{2L}\cos\left(\dfrac{a_{\rho t} \pi  t}{L}\right) +\\
&-\dfrac{1}{\gamma-1}\dfrac{ a_{pt}\pi p_t }{L}\sin\left(\dfrac{a_{pt} \pi  t}{L}\right)+\\
&-\dfrac{ a_{ut}\pi u_t \Rho_3 \U_3 }{L}\sin\left(\dfrac{a_{ut} \pi  t}{L}\right)+\\
&+\dfrac{ a_{vt}\pi v_t \Rho_3 \V_3 }{L}\cos\left(\dfrac{a_{vt} \pi t}{L}\right)+\\
&-\dfrac{ a_{wt}\pi w_t \Rho_3 \W_3 }{L}\sin\left(\dfrac{a_{wt} \pi  t}{L}\right).
 \end{split}
\end{equation}

\section{Comments}

The complexity, and consequently length, of the source terms increase with both dimension and physics handled by the governing equations. In some cases, such as the 3D transient energy equation, the final expression for $Q_{e_t}$ may reach 363,000 characters, including parenthesis and mathematical operators, prior to factorization.

Applying commands in order to simplify such extensive expression is challenging even with a very good machine; thus, a suitable alternative to this issue is to simplify the equation by dividing it into a combination of sub-operators handling different physical phenomena. Then, each one of the operators may be applied to the manufactured solutions individually and the resulting source terms are combined back to represent the source term for the original equation.


For instance, instead of writing the 3D transient Euler energy equation using one single operator~$\Lo$:

\begin{equation}
 \label{eq:ns2d_04}
\begin{split}
\Lo= \Diff{\rho e_t}{t}&+\nabla \cdot (\rho\mathbf{u}e_t)+ \nabla\cdot(p  \mathbf{u}) ,
%\Diff{\rho e_t}{t}& + \nabla \cdot (\rho \mathbf{u} H)+\nabla\cdot \mathbf{q} - \nabla \cdot (\bv{\tau u})
\end{split}
\end{equation}
to then be used in the MMS, let equation (\ref{eq:ns2d_04}) be written in three distinct operators, according to their physical meaning:
\begin{equation*}
 \begin{split}
  \Lo_1&=\Diff{\rho e_t}{t} ,\\
  \Lo_2&=\nabla \cdot (\rho\mathbf{u}e_t),\\
  \Lo_3&= \nabla\cdot(p  \mathbf{u}),\\
   \end{split}
\end{equation*}
where $\Lo_1$ denotes the rate of accumulation of inertial and kinetic energy, $\Lo_2$ is the net rate of internal and kinetic energy increase by convection,  $\Lo_3$ is the rate of work done on the fluid by external body forces. Naturally:
$$\Lo=\Lo_1+\Lo_2+\Lo_3.$$



 After the application of $\Lo_1$,  $\Lo_2$ and $\Lo_3$, the corresponding sub-source terms are also simplified, factorized and sorted. Then, the final factorized version is checked against the original one, to assure that not human error has been introduced.  This strategy allowed the original  363,000 character-long  expression for $Q_{e_t}$ to be reduced to less than 5,000, and expressed in Equation (\ref{eq:euler_3d_e}).


\subsection{Boundary Conditions}
Additionally to verifying code capability of solving the governing equations accurately in the interior of the domain of interest, one may also verify the software's capability of correctly imposing boundary conditions. Therefore, the gradients of the  analytical solutions (\ref{eq:manufactured01}) have been calculated and translated into $C$ codes. For the 3D case, they~are:
\begin{equation*}
\nabla  \rho= \left[ \begin{array}{c}
 \dfrac{  a_{\rho x}  \pi \rho_x}{L} \cos\left( \dfrac{ a_{\rho x}  \pi  x}{L}\right)\vspace{5pt} \\
-\dfrac{  a_{\rho y}  \pi \rho_y}{L} \sin\left( \dfrac{ a_{\rho y}  \pi  y}{L}\right)\vspace{5pt}\\
 \dfrac{  a_{\rho z}  \pi \rho_z}{L}  \cos\left( \dfrac{ a_{\rho z}  \pi  z}{L}\right)
\end{array} \right],
\qquad
\nabla p = \left[ \begin{array}{c}
- \dfrac{  a_{px}  \pi p_x}{L} \sin\left( \dfrac{ a_{px}  \pi  x}{L}\right)\vspace{5pt}\\
  \dfrac{  a_{py}  \pi p_y}{L} \cos\left( \dfrac{ a_{py}  \pi  y}{L}\right) \vspace{5pt}\\
- \dfrac{  a_{pz}  \pi p_z}{L} \sin\left( \dfrac{ a_{pz}  \pi  z}{L}\right)
\end{array} \right],
\quad
\nabla u = \left[ \begin{array}{c}
  \dfrac{  a_{ux}  \pi u_x}{L} \cos\left( \dfrac{ a_{ux}  \pi  x}{L}\right)\vspace{5pt}\\
 -   \dfrac{  a_{uy}  \pi u_y}{L} \sin\left( \dfrac{ a_{uy}  \pi  y}{L}\right)\vspace{5pt}\\
 -   \dfrac{  a_{uz}  \pi u_z}{L} \sin\left( \dfrac{ a_{uz}  \pi  z}{L}\right)
\end{array} \right],
\end{equation*}
\begin{equation*}
\nabla  v= \left[ \begin{array}{c}
-  \dfrac{  a_{vx}  \pi v_x}{L}  \sin\left( \dfrac{ a_{vx}  \pi  x}{L}\right)\vspace{5pt}\\
    \dfrac{  a_{vy}  \pi v_y}{L} \cos\left( \dfrac{ a_{vy}  \pi  y}{L}\right)\vspace{5pt}\\
   \dfrac{  a_{vz}  \pi v_z }{L} \cos\left( \dfrac{ a_{vz}  \pi  z}{L}\right)
\end{array} \right]
\quad\mbox{and}\quad
\nabla w = \left[ \begin{array}{c}
\dfrac{  a_{wx}  \pi  w_x}{L} \cos\left( \dfrac{ a_{wx}  \pi  x}{L}\right)\vspace{5pt}\\
  \dfrac{  a_{wy}  \pi w_y}{L}  \cos\left( \dfrac{ a_{wy}  \pi  y}{L}\right)\vspace{5pt} \\
 - \dfrac{  a_{wz}  \pi w_z}{L}  \sin\left( \dfrac{ a_{wz}  \pi  z}{L}\right)
\end{array} \right]
\end{equation*}


\subsection{C Files}
The $C$ files for both source terms and gradients of the  manufactured solutions are:
\begin{itemize}
 \item \texttt{Euler\_1d\_transient\_codes.C},
 \item \texttt{Euler\_1d\_manuf\_solutions\_grad\_code.C}
 \item \texttt{Euler\_2d\_transient\_e\_code.C}
 \item \texttt{Euler\_2d\_transient\_rho\_code.C}
 \item \texttt{Euler\_2d\_transient\_u\_code.C}
 \item \texttt{Euler\_2d\_transient\_v\_code.C}
 \item \texttt{Euler\_2d\_manuf\_solutions\_grad\_code.C}
 \item \texttt{Euler\_3d\_transient\_e\_code.C}
 \item \texttt{Euler\_3d\_transient\_rho\_code.C}
 \item \texttt{Euler\_3d\_transient\_u\_code.C}
 \item \texttt{Euler\_3d\_transient\_v\_code.C}
 \item \texttt{Euler\_3d\_transient\_w\_code.C}
 \item \texttt{Euler\_3d\_manuf\_solutions\_grad\_code.C}
\end{itemize}

For instance, automatically generated C file for the source term for the 3D total energy source term $Q_{e_t}$~is:
\begin{small}
\begin{verbatim}
#include <math.h>

double SourceQ_e (double x, double y, double z, double t)
{
  double Q_e_t;
  double RHO; double P; double U; double V; double W;

  RHO = rho_0 + rho_x * sin(a_rhox * PI * x / L) + rho_y * cos(a_rhoy * PI * y / L)
    + rho_z * sin(a_rhoz * PI * z / L) + rho_t * sin(a_rhot * PI * t / L);
  P = p_0 + p_x * cos(a_px * PI * x / L) + p_y * sin(a_py * PI * y / L) + p_z * cos(a_pz * PI * z / L)
    + p_t * cos(a_pt * PI * t / L);
  U = u_0 + u_x * sin(a_ux * PI * x / L) + u_y * cos(a_uy * PI * y / L) + u_z * cos(a_uz * PI * z / L)
    + u_t * cos(a_ut * PI * t / L);
  V = v_0 + v_x * cos(a_vx * PI * x / L) + v_y * sin(a_vy * PI * y / L) + v_z * sin(a_vz * PI * z / L)
    + v_t * sin(a_vt * PI * t / L);
  W = w_0 + w_x * sin(a_wx * PI * x / L) + w_y * sin(a_wy * PI * y / L) + w_z * cos(a_wz * PI * z / L)
    + w_t * cos(a_wt * PI * t / L);

  Q_e_t = -a_px * PI * p_x * U * Gamma * sin(a_px * PI * x / L) / (Gamma - 0.1e1) / L
    + a_py * PI * p_y * V * Gamma * cos(a_py * PI * y / L) / (Gamma - 0.1e1) / L
    - a_pz * PI * p_z * W * Gamma * sin(a_pz * PI * z / L) / (Gamma - 0.1e1) / L
    - a_ut * PI * u_t * RHO * U * sin(a_ut * PI * t / L) / L
    + a_vt * PI * v_t * RHO * V * cos(a_vt * PI * t / L) / L
    - a_wt * PI * w_t * RHO * W * sin(a_wt * PI * t / L) / L
    + (U * U + V * V + W * W) * a_rhox * PI * rho_x * U * cos(a_rhox * PI * x / L) / L / 0.2e1
    - (U * U + V * V + W * W) * a_rhoy * PI * rho_y * V * sin(a_rhoy * PI * y / L) / L / 0.2e1
    + (U * U + V * V + W * W) * a_rhoz * PI * rho_z * W * cos(a_rhoz * PI * z / L) / L / 0.2e1
    + (U * U + V * V + W * W) * a_rhot * PI * rho_t * cos(a_rhot * PI * t / L) / L / 0.2e1
    - a_pt * PI * p_t * sin(a_pt * PI * t / L) / (Gamma - 0.1e1) / L
    + (0.3e1 * a_ux * u_x * cos(a_ux * PI * x / L) + a_vy * v_y * cos(a_vy * PI * y / L)
      - a_wz * w_z * sin(a_wz * PI * z / L)) * PI * RHO * U * U / L / 0.2e1
    - (a_uy * u_y * sin(a_uy * PI * y / L) + a_vx * v_x * sin(a_vx * PI * x / L)) * PI * RHO * U * V / L
    + (-a_uz * u_z * sin(a_uz * PI * z / L) + a_wx * w_x * cos(a_wx * PI * x / L)) * PI * RHO * U * W / L
    + (a_ux * u_x * cos(a_ux * PI * x / L) + 0.3e1 * a_vy * v_y * cos(a_vy * PI * y / L)
      - a_wz * w_z * sin(a_wz * PI * z / L)) * PI * RHO * V * V / L / 0.2e1
    + (a_vz * v_z * cos(a_vz * PI * z / L) + a_wy * w_y * cos(a_wy * PI * y / L)) * PI * RHO * V * W / L
    + (a_ux * u_x * cos(a_ux * PI * x / L) + a_vy * v_y * cos(a_vy * PI * y / L)
        - 0.3e1 * a_wz * w_z * sin(a_wz * PI * z / L)) * PI * RHO * W * W / L / 0.2e1
    + (a_ux * u_x * cos(a_ux * PI * x / L) + a_vy * v_y * cos(a_vy * PI * y / L)
        - a_wz * w_z * sin(a_wz * PI * z / L)) * PI * P * Gamma / (Gamma - 0.1e1) / L;
  return(Q_e_t);
}
\end{verbatim}
\end{small}
%---------------------------------------------------------------------------------------------------------
\bibliographystyle{chicago} 
\bibliography{../../MMS_bib}

\end{document}

