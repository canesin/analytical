\documentclass[10pt,reqno]{amsart}

\usepackage{accents}
\usepackage{amsfonts}
\usepackage{amsmath}
\usepackage{amssymb}
\usepackage{fullpage}
\usepackage{hyperref}
\usepackage{listings}
\usepackage{multicol}
\usepackage{siunitx}
\usepackage{verbatim}

\newcommand{\tensor}[1]{\accentset{\leftrightarrow}{#1}}

\lstset{ %
basicstyle=\scriptsize,         % the size of the fonts that are used for the code
numbers=left,                   % where to put the line-numbers
numberstyle=\tiny,              % the size of the fonts that are used for the line-numbers
stepnumber=5,                   % the step between two line-numbers. If it's 1, each line
                                % will be numbered
numbersep=5pt,                  % how far the line-numbers are from the code
frame=single,                   % adds a frame around the code
showstringspaces=false          % underline spaces within strings
}

\begin{document}

\title{
    A transient manufactured solution for the compressible Navier--Stokes
    equations with a power law viscosity
}
\author{Rhys Ulerich}

\begin{abstract}
A time-varying manufactured solution is presented for the compressible
Navier--Stokes equations under the assumption of a constant Prandtl number,
Newtonian perfect gas obeying a power law viscosity.  The solution form
includes waveforms with adjustable phase offsets and mixed partial derivatives
to bolster code coverage.  Temperature, rather than pressure, is selected to
have a simple analytic form to aid verifying codes having temperature-based
boundary conditions.  Suggested verification tests for isothermal channel and
flat plate codes are provided.  A C++ implementation of the manufactured
solution and the forcing it requires is provided.  Tests are included to ensure
the implementation matches the solution form to within acceptable floating
point loss.
\end{abstract}

\maketitle

\section{Mathematical model}

Under assumptions of
\begin{itemize}
\item a Newtonian, compressible, perfect gas;
\item the fluid's first viscosity obeys a power law in temperature;
\item the second viscosity is a constant multiple of the first viscosity;
\item Fourier's law of heat conduction; and
\item a constant Prandtl number
\end{itemize}
the dimensional Navier--Stokes equations may take the form
\begin{subequations}
\label{eq:model}
\begin{align}
  \frac{\partial}{\partial{}t}\rho
&=
  - \nabla\cdot\rho{}\vec{u}
  + Q_{\rho}
  \\
  \frac{\partial{}}{\partial{}t}\rho{}\vec{u}
&=
  - \nabla\cdot(\vec{u}\otimes{}\rho{}\vec{u})
  -\nabla{} p
  + \nabla\cdot{} \tensor{\tau}
  + \vec{Q}_{\rho{}u}
  \\
  \frac{\partial}{\partial{}t} \rho{}e
&=
  - \nabla\cdot{}\rho{}e\vec{u}
  - \nabla\cdot{} p \vec{u}
  - \nabla\cdot{} \vec{q}
  + \nabla\cdot{}\tensor{\tau} \vec{u}
  + Q_{\rho{}e}
\intertext{
  with the auxiliary relations
}
  p &=   \left(\gamma-1\right)\left(\rho{}e
       - \rho\frac{\vec{u}\cdot{}\vec{u}}{2} \right)
  \\
  T &= \frac{p}{\rho{}R}
  \\
  \mu &= \mu_{r} \left( \frac{T}{T_{r}} \right)^{\beta}
  \\
  \lambda &= \frac{\lambda_{r}}{\mu_{r}} \mu
  \\
  \tensor{\tau}
       &=   \mu \left( \nabla{}\vec{u} + {\nabla{}\vec{u}}^{\mathsf{T}} \right)
          + \lambda \left( \nabla\cdot{}\vec{u} \right) I
  \\
  \vec{q} &= - \frac{\kappa_{r}}{\mu_{r}} \mu \nabla{} T
\end{align}
\end{subequations}
where all symbols have their customary interpretations.  Note $e$ denotes the
specific total energy and that later we will refer to the components of
$\vec{u}$ as the scalars $u$, $v$, and $w$.  The free constants in the model
are the ratio of specific heats $\gamma$, the gas constant $R$, the viscosity
power law exponent $\beta$, and the reference properties $\mu_r$, $T_r$,
$\kappa_r$, and $\lambda_r$.  One fixes $\kappa_r$ when choosing the Prandtl
number $\mbox{Pr}$ because $\kappa_r = \frac{\gamma R
\mu_r}{\left(\gamma-1\right)\mbox{Pr}}$.  Selecting $\lambda_r = - 2 \mu_r / 3$
recovers Stokes' hypothesis.  The thermal conductivity $\kappa$ does not appear
in the above equations as our constant Prandtl number assumption and the
observation that $\kappa$ increases with $\mu$ implies $\mu/\mu_r =
\kappa/\kappa_r$.  The arbitrary forcing terms $Q_{\rho}$, $\vec{Q}_{\rho{}u}$,
and $Q_{\rho{}e}$ will be used to obtain the desired manufactured solution.


\section{Manufactured solution}
\label{sec:solution}

For all $\phi\in\left\{\rho, u, v, w, T\right\}$ we select analytical solutions
of the form
\begin{alignat}{20}
\label{eq:solution}
  \phi\!\left(x, y, z, t\right)
  &= &&a_{\phi{}0}  &&          &&             &&               &&  &&            &&       &&          &&             &&                &&  &&            &&        &&\cos\Bigl(&&f_{\phi{}0 } &&t &&+ &&g_{\phi{}0 }&&\Bigr)       \\
  &+ &&a_{\phi{}x } &&\cos\Bigl(&&b_{\phi{}x } &&2\pi x L_x^{-1}&&+ &&c_{\phi{}x }&&\Bigr) &&          &&             &&                &&  &&            &&        &&\cos\Bigl(&&f_{\phi{}x } &&t &&+ &&g_{\phi{}x }&&\Bigr) \notag\\
  &+ &&a_{\phi{}xy} &&\cos\Bigl(&&b_{\phi{}xy} &&2\pi x L_x^{-1}&&+ &&c_{\phi{}xy}&&\Bigr) &&\cos\Bigl(&&d_{\phi{}xy} &&2\pi y L_y^{-1} &&+ &&e_{\phi{}xy}&&\Bigr)  &&\cos\Bigl(&&f_{\phi{}xy} &&t &&+ &&g_{\phi{}xy}&&\Bigr) \notag\\
  &+ &&a_{\phi{}xz} &&\cos\Bigl(&&b_{\phi{}xz} &&2\pi x L_x^{-1}&&+ &&c_{\phi{}xz}&&\Bigr) &&\cos\Bigl(&&d_{\phi{}xz} &&2\pi z L_z^{-1} &&+ &&e_{\phi{}xz}&&\Bigr)  &&\cos\Bigl(&&f_{\phi{}xz} &&t &&+ &&g_{\phi{}xz}&&\Bigr) \notag\\
  &+ &&a_{\phi{}y } &&\cos\Bigl(&&b_{\phi{}y } &&2\pi y L_y^{-1}&&+ &&c_{\phi{}y }&&\Bigr) &&          &&             &&                &&  &&            &&        &&\cos\Bigl(&&f_{\phi{}y } &&t &&+ &&g_{\phi{}y }&&\Bigr) \notag\\
  &+ &&a_{\phi{}yz} &&\cos\Bigl(&&b_{\phi{}yz} &&2\pi y L_y^{-1}&&+ &&c_{\phi{}yz}&&\Bigr) &&\cos\Bigl(&&d_{\phi{}yz} &&2\pi z L_z^{-1} &&+ &&e_{\phi{}yz}&&\Bigr)  &&\cos\Bigl(&&f_{\phi{}yz} &&t &&+ &&g_{\phi{}yz}&&\Bigr) \notag\\
  &+ &&a_{\phi{}z } &&\cos\Bigl(&&b_{\phi{}z } &&2\pi z L_z^{-1}&&+ &&c_{\phi{}z }&&\Bigr) &&          &&             &&                &&  &&            &&        &&\cos\Bigl(&&f_{\phi{}z } &&t &&+ &&g_{\phi{}z }&&\Bigr) \notag
\end{alignat}
where $a$, $b$, $c$, $d$, $e$, $f$, and $g$ are constant coefficient
collections indexed by $\phi$ and one or more directions.  To aid in providing
reusable, physically realizable coefficients for Cartesian domains of arbitrary
size, domain extents $L_x$, $L_y$, $L_z$ have been introduced.  Partial
derivatives $\phi_{t }$, $\phi_{x }$, $\phi_{y }$, $\phi_{z }$, $\phi_{xx}$,
$\phi_{xy}$, $\phi_{xz}$, $\phi_{yy}$, $\phi_{yz}$, and $\phi_{zz}$ may be
computed directly from the chosen solutions.

Though they increase the solution's complexity significantly, mixed partial
spatial derivatives are included to improve code coverage.  Each term has an
adjustable amplitude, frequency, and phase for all spatial dimensions.  Cosines
were chosen so all terms can be ``turned off'' by employing zero coefficients.
It is suggested that users gradually ``turn on'' the more complicated features
of the solution (i.e. use non-zero coefficients) after ensuring simpler usage
has been successful.

The Python-based computer algebra system SymPy (\url{http://sympy.org})
can both compute the derivatives and output C code for computing these
values at some $x$, $y$, $z$, and $t$:
\lstinputlisting[language=Python]{soln.py}

\section{Forcing terms}
\label{sec:forcing}

The solutions given by equation~\eqref{eq:solution} may be plugged into
equation~\eqref{eq:model} and solved for the forcing terms $Q_{\rho}$,
$\vec{Q}_{\rho{}u}$, and $Q_{\rho{}e}$.  However, solving for these complete
terms entirely within the context of a computer algebra system causes an
unwieldy explosion of terms.  As the fully expanded forcing terms are too
large to be usable in any meaningful way, they are not shown.

Instead, starting from the solution and its the analytic derivatives, we use
basic calculus followed by algebraic operations performed in floating point to
obtain the necessary forcing \emph{at runtime}.  The errors arising in this
process behave like standard floating point truncation issues.  Many of the
computations are independent of the constitutive relations used and could be
employed for other manufactured solutions.

Computing the forcing terms looks as follows:
\lstinputlisting[language=python]{forcing.py}

\section{Suggested coefficients for isothermal channels and flat plates}
\label{sec:suggest}

Employing the manufactured solution requires fixing the more than two hundred
coefficients appearing in equations \eqref{eq:model} and \eqref{eq:solution}.
Selecting usable values is not difficult but it can be time consuming.  We
therefore present reasonable coefficient choices for testing channel and flat
plate codes.

In both geometries the streamwise, wall-normal, and spanwise directions are
labeled $x$, $y$, and $z$ respectively.  Both $x$ and $z$ are periodic while
$y\in\left\{0,L_y\right\}$ is not.  Transient tests should likely take place
within the duration $0\leq{}t\leq{}1/10$ seconds as the time phase offsets
(e.g. $g_{Tyz}$) have been chosen for appreciable transients to occur
throughout this time window.

For isothermal channel flow code verification we recommend testing using
\begin{equation*}
  b_{\rho{}y} =
  b_{u{}y}    =
  b_{v{}y}    =
  b_{w{}y}    =
  b_{T{}y}    = \frac{1}{2}
\end{equation*}
and the coefficients given in Table~\ref{tbl:auxcoeff}.  With these choices
the manufactured solution satisfies isothermal, no-slip conditions at $y = 0,
L_y$.  For isothermal flat plate code verification we recommend testing using
\begin{equation*}
  b_{\rho{}y} =
  b_{u{}y}    =
  b_{v{}y}    =
  b_{w{}y}    =
  b_{T{}y}    = \frac{1}{4}
\end{equation*}
and the coefficients given in Table~\ref{tbl:auxcoeff}.  With these choices
the manufactured solution satisfies an isothermal, no-slip condition at $y =
0$.

\section{Reference Implementation}

A templated, precision-agnostic C++ implementation for evaluating the
manufactured solution and its associated forcing is included.  Also included is
a high precision test case which ensures the implementation computes exactly
what is described in this documentation.  The test can do so because it
executes the same Python source files used to generate this document.

One can ensure the implementation files match this documentation by comparing
against the following MD5 checksums:

\verbatiminput{CHECKSUMS}

More details on the implementation and tests can be found in the solution's
README:

\verbatiminput{README}

\begin{table}[p]
\allowdisplaybreaks
\begin{multicols}{3}
\begin{small}
\begin{align*}
\gamma &= \num{1.4} \\
R &= 287 \\
\beta &= 2 / 3 \\
\mu_r &= \num{1.852e-5} \\
T_r &= 300 \\
\kappa_r & = \frac{\gamma R \mu_r}{(\gamma - 1) \mbox{Pr}}
             \text{ with }\mbox{Pr} = 7/10 \\
\lambda_r & = - 2 \mu_r / 3 \\
L_x &= 4 \pi \\
L_y &= 2 \\
L_z &= 4 \pi / 3 \\
\intertext{}
a_{\rho{}0} &= 1 \\
a_{\rho{}xy} &= 1 / 11 \\
b_{\rho{}xy} &= 3 \\
d_{\rho{}xy} &= 3 \\
f_{\rho{}xy} &= 3 \\
g_{\rho{}xy} &= \pi / 4 \\
a_{\rho{}y} &= 1 / 7 \\
b_{\rho{}y} &= \text{\emph{see \textsection{}~\ref{sec:suggest}}} \\
f_{\rho{}y} &= 1 \\
g_{\rho{}y} &= \pi / 4 - 1 / 20 \\
a_{\rho{}yz} &= 1 / 31 \\
b_{\rho{}yz} &= 2 \\
d_{\rho{}yz} &= 2 \\
f_{\rho{}yz} &= 2 \\
g_{\rho{}yz} &= \pi / 4 + 1 / 20 \\
\intertext{}
a_{uxy} &= 53 / 37 \\
b_{uxy} &= 3 \\
c_{uxy} &= - \pi / 2 \\
d_{uxy} &= 3 \\
e_{uxy} &= - \pi / 2 \\
f_{uxy} &= 3 \\
g_{uxy} &= \pi / 4 \\
a_{uy} &= 53 \\
b_{uy} &= \text{\emph{see \textsection{}~\ref{sec:suggest}}} \\
c_{uy} &= -\pi / 2 \\
f_{uy} &= 1 \\
g_{uy} &= \pi / 4 - 1 / 20 \\
a_{uyz} &= 53 / 41 \\
b_{uyz} &= 2 \\
c_{uyz} &= - \pi / 2 \\
d_{uyz} &= 2 \\
e_{uyz} &= - \pi / 2 \\
f_{uyz} &= 2 \\
g_{uyz} &= \pi / 4 + 1 / 20 \\
\intertext{}
a_{vxy} &= 3 \\
b_{vxy} &= 3 \\
c_{vxy} & = - \pi / 2 \\
d_{vxy} & = 3         \\
e_{vxy} & = - \pi / 2 \\
f_{vxy} &= 3 \\
g_{vxy} &= \pi / 4 \\
a_{vy} &= 2 \\
b_{vy} &= \text{\emph{see \textsection{}~\ref{sec:suggest}}} \\
c_{vy} &= - \pi / 2 \\
f_{vy} &= 1 \\
g_{vy} &= \pi / 4 - 1 / 20 \\
a_{vyz} &= 5 \\
b_{vyz} &= 2 \\
c_{vyz} &= -\pi / 2 \\
d_{vyz} &= 2 \\
e_{vyz} &= -\pi / 2 \\
f_{vyz} &= 2 \\
g_{vyz} &= \pi / 4 + 1 / 20 \\
\intertext{}
a_{wxy} &= 11 \\
b_{wxy} &= 3 \\
c_{wxy} &= -\pi / 2 \\
d_{wxy} &= 3 \\
e_{wxy} &= -\pi / 2 \\
f_{wxy} &= 3 \\
g_{wxy} &= \pi / 4 \\
a_{wy} &= 7 \\
b_{wy} &= \text{\emph{see \textsection{}~\ref{sec:suggest}}} \\
c_{wy} &= - \pi / 2 \\
f_{wy} &= 1 \\
g_{wy} &= \pi / 4 - 1 / 20 \\
a_{wyz} &= 13 \\
b_{wyz} &= 2 \\
c_{wyz} &= - \pi / 2 \\
d_{wyz} &= 2 \\
e_{wyz} &= - \pi / 2 \\
f_{wyz} &= 2 \\
g_{wyz} &= \pi / 4 + 1 / 20 \\
\intertext{}
a_{T0} &= 300 \\
a_{Txy} &= 300 / 17 \\
b_{Txy} &= 3 \\
c_{Txy} &= - \pi / 2 \\
d_{Txy} &= 3 \\
e_{Txy} &= - \pi / 2 \\
f_{Txy} &= 3 \\
g_{Txy} &= \pi / 4 \\
a_{Ty} &= 300 / 13 \\
b_{Ty} &= \text{\emph{see \textsection{}~\ref{sec:suggest}}} \\
c_{Ty} &= - \pi / 2 \\
f_{Ty} &= 1 \\
g_{Ty} &= \pi / 4 - 1 / 20 \\
a_{Tyz} &= 300 / 37 \\
b_{Tyz} &= 2 \\
c_{Tyz} &= - \pi / 2 \\
d_{Tyz} &= 2 \\
e_{Tyz} &= - \pi / 2 \\
f_{Tyz} &= 2 \\
g_{Tyz} &= \pi / 4 + 1 / 20 \\
\end{align*}
\end{small}
\end{multicols}
\caption{Coefficient recommendations from section~\ref{sec:suggest}.
         Standard MKS units are implied with each value; e.g.
         $R$ is given in
         \si{\joule\per\kilo\gram\per\kelvin} and $\mu_r$ is given in
         \si{\pascal\second}.
         Unlisted coefficients should be set to zero.
         \label{tbl:auxcoeff}}
\end{table}

\end{document}
