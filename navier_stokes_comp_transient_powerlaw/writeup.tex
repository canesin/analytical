\documentclass[10pt,reqno]{amsart}

\usepackage{amsfonts}
\usepackage{amsmath}
\usepackage{amssymb}
\usepackage{fullpage}
\usepackage{hyperref}
\usepackage{listings}

\lstset{ %
basicstyle=\scriptsize,         % the size of the fonts that are used for the code
numbers=left,                   % where to put the line-numbers
numberstyle=\tiny,              % the size of the fonts that are used for the line-numbers
stepnumber=5,                   % the step between two line-numbers. If it's 1, each line
                                % will be numbered
numbersep=5pt,                  % how far the line-numbers are from the code
frame=single,                   % adds a frame around the code
showstringspaces=false          % underline spaces within strings
}

\begin{document}

\title{
    A transient manufactured solution for the compressible Navier--Stokes
    equations with a power law viscosity
}
\author{Rhys Ulerich}

\begin{abstract}
A time-varying manufactured solution is presented for the compressible
Navier--Stokes equations under the assumption of a constant Prandtl number,
Newtonian perfect gas obeying a power law viscosity.  The solution form
includes waveforms with adjustable phase offsets and mixed partial derivatives
to bolster code coverage.  Temperature, rather than pressure, is selected to
have a simple analytic form to aid verifying codes having temperature-based
boundary conditions.
\end{abstract}

\maketitle

\section{Mathematical model}

Under assumptions of
\begin{itemize}
\item a Newtonian, compressible, perfect gas;
\item the fluid's first viscosity obeys a power law in temperature;
\item the second viscosity is a constant multiple of the first viscosity; and
\item constant Prandtl number
\end{itemize}
the dimensional Navier--Stokes equations may take the form
\begin{subequations}
\label{eq:model}
\begin{align}
  \frac{\partial}{\partial{}t}\rho
&=
  - \nabla\cdot\rho{}\vec{u}
  + Q_{\rho}
  \\
  \frac{\partial{}}{\partial{}t}\rho{}\vec{u}
&=
  - \nabla\cdot(\vec{u}\otimes{}\rho{}\vec{u})
  -\nabla{} p
  + \nabla\cdot{} \tau
  + \vec{Q}_{\rho{}u}
  \\
  \frac{\partial}{\partial{}t} \rho{}e
&=
  - \nabla\cdot{}\rho{}eu
  + \nabla\cdot{} \frac{\kappa_{r}}{\mu_{r}} \mu \nabla{} T
  - \nabla\cdot{} p \vec{u}
  + \nabla\cdot{}\tau{} \vec{u}
  + Q_{\rho{}e}
\intertext{
  with the auxiliary relations
}
  p &=   \left(\gamma-1\right)\left(\rho{}e
       - \rho\frac{\vec{u}\cdot{}\vec{u}}{2} \right)
  \\
  T &= \frac{p}{\rho{}R}
  \\
  \mu &= \mu_{r} \left( \frac{T}{T_{r}} \right)^{\beta}
  \\
  \lambda &= \frac{\lambda_{r}}{\mu_{r}} \mu
  \\
  \tau &=   \mu \left( \nabla{}\vec{u} + {\nabla{}\vec{u}}^{\mathsf{T}} \right)
          + \lambda \left( \nabla\cdot{}\vec{u} \right) I
\end{align}
\end{subequations}
where all symbols have their customary interpretations.  Note $e$ denotes the
specific total energy and that later we will refer to the components of
$\vec{u}$ as the scalars $u$, $v$, and $w$.  The free constants in the model
are the ratio of specific heats $\gamma$, the gas constant $R$, the viscosity
power law exponent $\beta$, and the reference properties $\mu_r$, $T_r$,
$\kappa_r$, and $\lambda_r$.  One fixes $\kappa_r$ when choosing the Prandtl
number $\mbox{Pr}$ because $\kappa_r = \frac{\gamma R
\mu}{\left(\gamma-1\right)\mbox{Pr}}$.  Selecting $\lambda_r = - 2 \mu_r / 3$
recovers Stokes' hypothesis.  The thermal conductivity $\kappa$ does not appear
in the above equations as our constant Prandtl number assumption and the
observation that $\kappa$ increases with $\mu$ implies $\mu/\mu_r =
\kappa/\kappa_r$.  The arbitrary forcing terms $Q_{\rho}$, $\vec{Q}_{\rho{}u}$,
and $Q_{\rho{}e}$ will be used to obtain the desired manufactured solution.


\section{Manufactured solution}

For all $\phi\in\left\{\rho, u, v, w, T\right\}$ we select analytical solutions
of the form
\begin{alignat}{20}
\label{eq:solution}
  \phi\!\left(x, y, z, t\right)
&= &&a_{\phi{}0}  &&     &&             &&  &&  &&            &&  &&     &&             &&  &&  &&            &&   &&\cos(&&f_{\phi{}0 } &&t &&+ &&g_{\phi{}0 }&&)       \\
&+ &&a_{\phi{}x } &&\cos(&&b_{\phi{}x } &&x &&+ &&c_{\phi{}x }&&) &&     &&             &&  &&  &&            &&   &&\cos(&&f_{\phi{}x } &&t &&+ &&g_{\phi{}x }&&) \notag\\
&+ &&a_{\phi{}xy} &&\cos(&&b_{\phi{}xy} &&x &&+ &&c_{\phi{}xy}&&) &&\cos(&&d_{\phi{}xy} &&y &&+ &&e_{\phi{}xy}&&)  &&\cos(&&f_{\phi{}xy} &&t &&+ &&g_{\phi{}xy}&&) \notag\\
&+ &&a_{\phi{}xz} &&\cos(&&b_{\phi{}xz} &&x &&+ &&c_{\phi{}xz}&&) &&\cos(&&d_{\phi{}xz} &&z &&+ &&e_{\phi{}xz}&&)  &&\cos(&&f_{\phi{}xz} &&t &&+ &&g_{\phi{}xz}&&) \notag\\
&+ &&a_{\phi{}y } &&\cos(&&b_{\phi{}y } &&y &&+ &&c_{\phi{}y }&&) &&     &&             &&  &&  &&            &&   &&\cos(&&f_{\phi{}y } &&t &&+ &&g_{\phi{}y }&&) \notag\\
&+ &&a_{\phi{}yz} &&\cos(&&b_{\phi{}yz} &&y &&+ &&c_{\phi{}yz}&&) &&\cos(&&d_{\phi{}yz} &&z &&+ &&e_{\phi{}yz}&&)  &&\cos(&&f_{\phi{}yz} &&t &&+ &&g_{\phi{}yz}&&) \notag\\
&+ &&a_{\phi{}z } &&\cos(&&b_{\phi{}z } &&z &&+ &&c_{\phi{}z }&&) &&     &&             &&  &&  &&            &&   &&\cos(&&f_{\phi{}z } &&t &&+ &&g_{\phi{}z }&&) \notag
\end{alignat}
where $a$, $b$, $c$, $d$, $e$, $f$, and $g$ are fixed constant collections
indexed by $\phi$ and one or more directions.  Partial derivatives $\phi_{t }$,
$\phi_{x }$, $\phi_{y }$, $\phi_{z }$, $\phi_{xx}$, $\phi_{xy}$, $\phi_{xz}$,
$\phi_{yy}$, $\phi_{yz}$, and $\phi_{zz}$ may be computed directly from the
chosen solutions.

Though they increase the solution's complexity significantly, mixed partial
spatial derivatives are included to improve code coverage.  Each term has an
adjustable amplitude and phase for all involved dimensions.  Cosines were
chosen so terms can be ``turned off'' by employing zero coefficients. It is
suggested that users gradually ``turn on'' the more complicated features of the
solution (i.e. use non-zero coefficients) after ensuring simpler usage has been
successful.

The Python-based computer algebra system SymPy (\url{http://sympy.org})
can both compute the derivatives and output C code for computing these
values at some $x$, $y$, $z$, and $t$:
\lstinputlisting[language=Python]{soln.py}

\section{Forcing terms}

The solutions given by equation~\eqref{eq:solution} may be plugged into
equation~\eqref{eq:model} and solved for the forcing terms $Q_{\rho}$,
$\vec{Q}_{\rho{}u}$, and $Q_{\rho{}e}$.  However, solving for these complete
terms entirely within the context of a computer algebra system causes an
unwieldy explosion of terms.

Instead, starting from the solution and its the analytic derivatives, we use
basic calculus and algebraic operations performed in floating point to obtain
the necessary forcing.  The errors arising in this process behave like standard
floating point truncation issues.  Many of the computations are independent of
the constitutive relations used and could be employed for other manufactured
solutions.

Computing the forcing terms looks as follows:
\lstinputlisting[language=python]{forcing.py}

\end{document}
