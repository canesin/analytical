\documentclass{article}
\usepackage{amsmath}
\usepackage{graphicx}
\title{\bf{Manufactured Solution for the Incompressible Navier-Stokes}}
\author{Nicholas Malaya \\ Nicholas Fitzsimmons \\ Robert D. Moser} \date{}

\begin{document}
\maketitle

Construct a 3D incompressible field. Define three functions: f(x), g(y),
h(z) satisfying boundary conditions. 

Define the three components of velocity as,
\begin{align}
u = a f(x)  g'(y)  h'(z) \\
v = b f'(x) g(y)   h'(z) \\
w = c f'(x) g'(y)  h(z) 
\end{align}
To satisfy continuity, $a+b+c=0$. We can include time dependence
$a=a(t)$ if we so desire.

\begin{equation}
 g(y) = \frac{1}{\delta + \text{sin}(k_y y)} \mbox{if } \beta > 1
\end{equation}

In order to provide high wavenumber content, choose f(x) as,
\begin{equation}
 f(x) =
  \begin{cases}
   \text{sin}(k_x x)  \\
   \frac{1}{\beta + \text{sin}(k_x x)} & \mbox{if } \beta > 1 \\
  \end{cases}
\end{equation}
Likewise choose the manufactured solution for the spanwise component of
velocity to be
\begin{equation}
 h(z) =
  \begin{cases}
   \text{sin}(k_z z)  \\
   \frac{1}{\gamma + \text{sin}(k_z z)} & \mbox{if } \gamma > 1 \\
  \end{cases}
\end{equation}
Finally, manufacture the pressure such that, 
\begin{equation}
 P = d f(x) g(y) h(z)
\end{equation}
Where d is just a scaling parameter. 
\newpage
\section{MMS}
\begin{verbatim}
  // NS equation residuals
  NumberArray<NDIM, Scalar> Q_rho_u = 
    raw_value(

	      // convective term
	      -divergence(U.outerproduct(U))

	      // pressure
	      - P.derivatives()

	      // dissipation
	      + nu * divergence(gradient(U)));

  return Q_rho_u[1];
\end{verbatim}

\end{document}