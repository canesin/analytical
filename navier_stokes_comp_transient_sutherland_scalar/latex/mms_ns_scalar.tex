\documentclass[10pt]{article}
\usepackage[utf8x]{inputenc}
\usepackage{amsmath}
\usepackage{geometry}
%\usepackage[mathcal]{euscript}
\geometry{ top=2.5cm, bottom=2cm, left=2cm, right=2cm}
\usepackage[authoryear]{natbib}
\usepackage{pdflscape}
%\geometry{papersize={216mm,330mm}, top=3cm, bottom=2.5cm, left=4cm,  right=2cm}

\newcommand{\D}{\partial}
\newcommand{\Diff}[2] {\dfrac{\partial( #1)}{\partial #2}}
\newcommand{\diff}[2] {\dfrac{\partial #1}{\partial #2}}
\newcommand{\bv}[1]{\ensuremath{\mbox{\boldmath$ #1 $}}}
\newcommand{\gv}[1]{\ensuremath{\mbox{\boldmath$ #1 $}}}% for vectors of Greek letters
\newcommand{\grad}[1]{\gv{\nabla} #1}
\newcommand{\Rho}{\,\mathtt{Rho}}
\newcommand{\PP}{\,\mathtt{P}}
\newcommand{\U}{\,\mathtt{U}}
\newcommand{\V}{\,\mathtt{V}}
\newcommand{\W}{\,\mathtt{W}}
\newcommand{\Lo}{\,\mathcal{L}}
\newcommand{\MU}{\,\mathtt{Mu}}%opening
\title{Manufactured Solution for the Compressible Steady Navier--Stokes Equations with Sutherland Viscosity and One Passive Scalar}
\author{Nicholas Malaya}

\begin{document}

\maketitle

\begin{abstract}
  Create an MMS for the 3D compressible navier stokes with a passive scalar 
  for verification of both compDNS and Suzerain. 
\end{abstract}

\section{Mathematical Model}
The conservation of mass, momentum, and total energy for a compressible steady viscous fluid may be written as:
\begin{equation}
 \label{eq:ns_01}
\nabla \cdot \left(\rho \bv{u}\right) = 0,
\end{equation}

\begin{equation}
 \label{eq:ns_02}
\nabla\cdot\left(\rho\bv{u}\bv{u}\right) = -\nabla p+  \nabla \cdot (\bv{\tau} ),
\end{equation}

\begin{equation}
 \label{eq:ns_03}
%\nabla \cdot (\rho\bv{u}e_t)+  \nabla\cdot(p  \bv{u})=0
\nabla\cdot\left(\rho \bv{u} H\right) =-   \nabla\cdot(p  \mathbf{u}) -\nabla\cdot \mathbf{q} +  \nabla \cdot (\bv{\tau} \cdot \mathbf{u}).
\end{equation}

Equations (\ref{eq:ns_01})--(\ref{eq:ns_03}) are known as Navier--Stokes equations and, $\rho$ is the density, $\bv{u}=(u,v,w)$ is the velocity in $x$, $y$ or $z$-direction, respectively,    and $p$ is the pressure. The total enthalpy, $H$, may be expressed in terms of the total energy per unit mass $e_t$, density, and pressure:
$$H = e_t + \dfrac{p}{\rho}.$$
For a calorically perfect gas, the Navier--Stokes equations are closed with two auxiliary relations for energy:
\begin{equation}
 \label{eq:ns_04}
e_t= e+\dfrac{\bv{u}\cdot \bv{u}}{2},\quad\mbox{and}\quad e=\dfrac{1}{\gamma -1}RT ,
\end{equation}
and with the ideal gas equation of state:
\begin{equation}
 \label{eq:ns_05}
p=\rho RT.
\end{equation}

The shear stress tensor is:
\begin{equation}
 \begin{array}{lll}
  \tau_{xx}= \dfrac{2}{3}  \mu \left( 2 \diff{u}{x} - \diff{v}{y} -\diff{w}{z} \right),
  &\tau_{xy}= \tau_{yx}=\mu \left( \diff{u}{y} + \diff{v}{x}\right),\\
  \tau_{yy}= \dfrac{2}{3}  \mu \left( 2 \diff{v}{y} - \diff{u}{x} -\diff{w}{z} \right),
  &\tau_{yz}= \tau_{zy}=\mu \left( \diff{v}{z} + \diff{w}{y}\right),\\
  \tau_{zz}= \dfrac{2}{3}  \mu \left( 2 \diff{w}{z} - \diff{u}{x} -\diff{v}{y} \right),
  &\tau_{xz}= \tau_{zx}=\mu \left( \diff{u}{z} + \diff{w}{x}\right),
 \end{array}
\end{equation}
where $\mu$ is the absolute viscosity. The heat flux vector $\mathbf{q}=(q_x,q_y,q_z)$ is given by:
\begin{equation}
 %\begin{split}
 % q_x &= - k \diff{T}{x}\\
%q_y &= - k \diff{T}{y}
% \end{split}
 q_x = - k \diff{T}{x}, \quad q_y = - k \diff{T}{y}, \quad \mbox{and} \quad q_z = - k \diff{T}{z}
 \end{equation}
where $k$ is the thermal conductivity, which can be determined by choosing the Prandtl number:
$$k= \dfrac{\gamma R \mu}{ (\gamma-1) \text{Pr}}.$$

\subsection{Sutherland viscosity model}

\citet{Sutherland1893} published a relationship between the absolute temperature of an ideal gas, $T$,  and its dynamic (absolute) viscosity, $\mu$. The model is based on the kinetic theory of ideal gases and an idealized intermolecular-force potential. The general equation is given as:
\begin{equation}
\label{eq:Sutherland01}
 \mu  =\dfrac{A_\mu \, T^{\frac{3}{2}}}{T+B_\mu}
\end{equation}
with
$$    A_\mu = \dfrac{\mu_\text{ref}}{T_{\text{ref}}^{\frac{3}{2}}}(T_\text{ref} + B_\mu), $$
%
where $B_\mu$ is the Sutherland temperature, $T_{\text{ref}}$ is a reference temperature,  and $\mu_\text{ref}$ is the viscosity at the reference temperature $T_\text{ref}$.

\end{document}
