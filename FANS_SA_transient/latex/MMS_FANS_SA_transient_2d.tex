\documentclass[10pt]{article}
\usepackage[utf8x]{inputenc}
\usepackage{amssymb, amsmath, amsfonts, amsthm, wasysym} % math
\usepackage{nicefrac}
\usepackage{geometry}
%\usepackage[mathcal]{euscript}
\geometry{ top=2.5cm, bottom=2cm, left=2cm, right=2cm}
\usepackage[authoryear]{natbib}
\usepackage{pdflscape}
%\geometry{papersize={216mm,330mm}, top=3cm, bottom=2.5cm, left=4cm,  right=2cm}

\newcommand{\D}{\partial}
\newcommand{\Diff}[2] {\dfrac{\partial( #1)}{\partial #2}}
\newcommand{\diff}[2] {\dfrac{\partial #1}{\partial #2}}
\newcommand{\bv}[1]{\ensuremath{\mbox{\boldmath$ #1 $}}}
\newcommand{\gv}[1]{\ensuremath{\mbox{\boldmath$ #1 $}}}% for vectors of Greek letters
\newcommand{\grad}[1]{\gv{\nabla} #1}
\newcommand{\Rho}{\,\mathtt{Rho}}
\newcommand{\PP}{\,\mathtt{P}}
\newcommand{\U}{\,\mathtt{U}}
\newcommand{\V}{\,\mathtt{V}}
\newcommand{\Nu}{\,\mathtt{Nu_{sa}}}
\newcommand{\Mu}{\,\mathtt{Mu}}
\newcommand{\Lo}{\,\mathcal{L}}
%\newcommand{\Pr}{\mbox{Pr}}

% commands I like
\newcommand{\mbb}[1]{\mathbb{#1}}
\newcommand{\mbf}[1]{\mathbf{#1}}
\newcommand{\sbf}[1]{\boldsymbol{#1}}
\newcommand{\mcal}[1]{\mathcal{#1}}
\newcommand{\mfk}[1]{\mathfrak{#1}}
\newcommand{\pp}[2]{\frac{\partial #1}{\partial #2}}
\newcommand{\dd}[2]{\frac{d #1}{d #2}}
\newcommand{\rarrow}{\rightarrow}
\newcommand{\Rarrow}{\Rightarrow}
\newcommand{\LRarrow}{\Leftrightarrow}
\newcommand{\jump}[1]{\llbracket #1 \rrbracket}
\newcommand{\avg}[1]{\{ #1 \}}
\def\etal{{\it et al.~}}
\newcommand{\vvvert}{|\kern-1pt|\kern-1pt|}
\newcommand{\enorm}[1]{\vvvert #1 \vvvert}
\newcommand{\ud}{\,\mathrm{d}}

\newcommand{\sa}{\nu_{\mathrm{sa}}}
\newcommand{\tsa}{\mathrm{sa}}
\newcommand{\brho}{\bar{\rho}}
\newcommand{\bp}{\bar{p}}
\newcommand{\bq}{\bar{q}}
\newcommand{\tu}{\tilde{u}}
\newcommand{\tv}{\tilde{v}}
\newcommand{\tS}{\tilde{S}}
\newcommand{\tE}{\tilde{E}}
\newcommand{\bmu}{\bar{\mu}}
\newcommand{\hh}{\tilde{h}}
%opening
\title{Manufactured Solution for the 2D Favre-Averaged Navier--Stokes Equations with Spalart-Allmaras turbulence model using Maple}
\author{Kemelli C. Estacio-Hiroms}

\begin{document}

\maketitle

\begin{abstract}
The Method of Manufactured Solutions is a valuable approach for code verification, providing means to verify how accurately the numerical method solves the partial differential equations of interest.
This document presents the source terms generated by the application of the Method of Manufactured Solutions on the 2D transient Favre-Averaged Navier--Stokes Equations with Spalart-Allmaras turbulence model using the analytical manufactured solutions for density, velocity and pressure presented by \citet{Roy2002}.
\end{abstract}





\section{Mathematical Model}

%-------------------------------------------------
Turbulent flows occur at high Reynolds numbers, when the inertia of the fluid overwhelms the viscosity of the fluid, causing the laminar flow motions to become unstable. Under these conditions, the flow is characterized by rapid fluctuations in pressure and velocity which are inherently three dimensional and unsteady. Turbulent flow is composed of large eddies that migrate across the flow generating smaller eddies as they go. These smaller eddies in turn generates smaller eddies until they become small enough that their energy is dissipated due to the presence of molecular viscosity.

In practice, the effect of this sensitivity
is to make the value of any flow quantity at any particular point in
time and space uncertain.  Thus, these quantities may be viewed as
random variables with associated probability density functions,
allowing the use of statistical techniques in the description and
analysis of the flow. Or, in other words, the full influence of the turbulent fluctuations on the mean flow must be modelled.

For flows with significant density variations it is possible to  capture the turbulent effects using the Favre averaged Navier-Stokes equations (FANS), together with the one-equation Spalart-Allmaras (SA) turbulent model \citep{Oliver2010}:

Mass conservation:
\begin{equation}\label{eq:ns01}
\pp{\bar{\rho}}{t} + \pp{}{x_i} (\bar{\rho}\tilde{u}_i) = 0, 
\end{equation}

Momentum conservation:
\begin{equation}\label{eq:ns02}
\pp{}{t} \left(\bar{\rho} \tilde{u}_i \right) + \pp{}{x_j} \left(\bar{\rho} \tilde{u}_j \tilde{u}_i  \right) = - \, \pp{\bar{p}}{x_i} + \pp{}{x_j}\left( 2 (\bar{\mu} + \mu_t) \tilde{S}_{ji} \right), \\
\end{equation}

Total energy conservation:
\begin{equation}\label{eq:ns03}
\pp{}{t} \left[ \bar{\rho} \left( \tilde{e} + \frac{1}{2} \tilde{u}_i \tilde{u}_i \right) \right] + \pp{}{x_j} \left[ \bar{\rho} \tilde{u}_j \left( \tilde{h} + \frac{1}{2} \tilde{u}_i \tilde{u}_i \right) \right] =  \pp{}{x_j} \left( 2 (\bar{\mu} + \mu_t) \tilde{S}_{ji} \tilde{u}_i \right) + \pp{}{x_j} \left[ \left( \frac{\bar{\mu}}{\Pr} + \frac{\mu_t}{\Pr_t} \right) \pp{\tilde{h}}{x_j} \right], \\
\end{equation}

Baseline compressible Spalart-Allmaras equation:
\begin{equation}\label{eq:ns04}
\pp{}{t}(\bar{\rho} \sa) + \pp{}{x_j} (\bar{\rho} \tilde{u}_j \sa) =  c_{b1} S_{\mathrm{sa}} \bar{\rho} \sa - c_{w1} f_w \bar{\rho} \left( \frac{\sa}{d} \right)^2 + \frac{1}{\sigma} \pp{}{x_k} \left[ (\bar{\mu} + \bar{\rho} \sa) \pp{\sa}{x_k} \right] + \frac{c_{b2}}{\sigma} \bar{\rho} \pp{\sa}{x_k} \pp{\sa}{x_k},
\end{equation}
%
where $[\tilde{\,\,}]$ denotes a Favre-averaging variable and $[\bar{\,\,}]$ denotes Reynolds averaging.

To close the equations, many additional relationships are
necessary---e.g., a constitutive relation for the viscous stress, an
equation of state, etc. In this work, the gas is considered calorically perfect and:
%
\begin{equation}
 \begin{split}\label{eq:closure}
&\bar{\mu} = \mu_0 \left( \frac{\tilde{T}}{T_0} \right)^{3/2} \frac{T_0 + S}{\tilde{T} + S}, \quad \tilde{S}_{ij} = \tilde{s}_{ij} - \frac{1}{3} \tilde{s}_{kk} \delta_{ij}, \quad \tilde{s}_{ij} = \frac{1}{2} \left( \pp{\tilde{u}_i}{x_j} + \pp{\tilde{u}_j}{x_i} \right), \\
&\bar{p} = \bar{\rho} R \tilde{T}, \quad \tilde{e} = c_v \tilde{T}, \quad \tilde{h} = c_p \tilde{T} = \tilde{e} + \frac{\bar{p}}{\bar{\rho}} \\
&\mu_t = \bar{\rho} \nu_t = \bar{\rho} \sa f_{v1}, \quad S_{\mathrm{sa}} = \Omega + \frac{\sa}{\kappa^2 d^2} f_{v2}, \quad \Omega = \sqrt{2 \tilde{\Omega}_{ij} \tilde{\Omega}_{ij} }, \quad \widetilde{\Omega}_{ij} = \frac{1}{2} \left( \pp{\tilde{u}_i}{x_j} - \pp{\tilde{u}_j}{x_i} \right), \\
&f_{v2} = 1 - \frac{\chi}{1 + \chi f_{v1}}, \quad f_{v1} = \frac{\chi^3}{\chi^3 + c_{v1}^3}, \quad \chi = \frac{\sa}{\tilde{\nu}}, \\
&f_w = g \left( \frac{1 + c_{w3}^6}{g^6 + c_{w3}^6} \right)^{1/6}, \quad g = r + c_{w2} \left( r^6 - r \right), \quad r = \frac{\sa}{S_{\mathrm{sa}} \kappa^2 d^2}.  
 \end{split}
\end{equation}
%
where $d$ is the distance to the nearest no slip wall.

The constants $c_v$ and $c_p$ are fluid properties.  The constants
$\mu_0$, $T_0$, $S$, are the calibration parameters appearing in
Sutherland's law, and $c_{b1}$, $c_{b2}$, $c_{v1}$, $\sigma$,
$c_{w1}$, $c_{w2}$, $c_{w3}$, and $\kappa$ are the SA model
calibration parameters.\\


\textbf{Note:} In this work the averaged absolute viscosity is considered constant, $\bmu=\texttt{constant}$ and the distance to the nearest no-slip wall is assumed to be infinite, $d\rightarrow\infty$.
Due to such assumptions, the simplifications carried out in the governing equations as well as in the closure expressions  are presented in Section \ref{NS+SA}.
%
% where here, $\bar{\rho}$ and $\bar{p}$ are Reynolds-averaged density and pressure, respectively,  $\tilde{u}_i$ is the Favre averaged velocity and $u_i''$ is the fluctuations due to turbulence. % It is assumed that $u_i=\tilde{u}_i+u_i''$.
% %
% %
% $\tilde{E}$ and $\tilde{H}$ are the Favre-averaged total energy per unit mass and the total enthalpy per unit mass, respectively. The variable $\bar{\tau}_{ji} $ is the instantaneous viscous stress tensor, $h$ is the specific enthalpy, and $\bar{q}_j$ is the laminar mean heat-flux vector. %The Favre averaging of the Navier -Stokes equations introduces higher-order velocity correlations that reflect the influence of the turbulence on the mean flow. The Favre averaged Reynolds stress tensor is related to the time-averaged product of the turbulent velocity fluctuations and is given by
% 
% Examining the Favre-averaged form, one can identify multiple effects
% of turbulence: the Reynolds stress, $-\overline{\rho u_j'' u_i''}$,
% augmenting the viscous stress; the Reynolds heat flux, $\overline{\rho
%   u_j'' h''}$, augmenting the heat flux; and a term that is referred
% to here as the ``turbulent transport and work''.  The turbulent
% transport and work term contains two effects: transport of TKE by
% turbulent velocity fluctuations ($-\frac{1}{2} \overline{\rho u_j''
%   u_i'' u_i''}$) and work done by the viscous stress due to turbulent
% velocity fluctuations ($\overline{\tau_{ji} u_i''}$).
% 
% 
% To close the equations, many additional relationships are
% necessary---e.g., a constitutive relation for the viscous stress, an
% equation of state, etc.  In this work, the gas is considered Newtonian and calorically perfect.
% % 
% % \begin{gather*}
% % \myred{\bar{\mu} = \mu_0 \left( \frac{\tilde{T}}{T_0} \right)^{3/2} \frac{T_0 + S}{\tilde{T} + S}}, \quad \tilde{S}_{ij} = \tilde{s}_{ij} - \frac{1}{3} \tilde{s}_{kk} \delta_{ij}, \quad \tilde{s}_{ij} = \frac{1}{2} \left( \pp{\tilde{u}_i}{x_j} + \pp{\tilde{u}_j}{x_i} \right), \\
% % \bar{p} = \bar{\rho} R \tilde{T}, \quad \tilde{E} = \tilde{e} + \frac{1}{2} \tilde{u}_i \tilde{u}_i + k, \quad \tilde{e} = c_v \tilde{T}, \quad \tilde{H} = \tilde{h} + \frac{1}{2} \tilde{u}_i \tilde{u}_i + k, \quad \tilde{h} = c_p \tilde{T} = \tilde{e} + \frac{\bar{p}}{\bar{\rho}}\, .
% % \end{gather*}
%-------------------------------------------------

\section{Manufactured Solution}

The Method of Manufactured Solutions (MMS) applied to Favre-Averaged Navier--Stokes equations with baseline compressible Spalart-Allmaras turbulence model consists in modifying Equations~(\ref{eq:ns01})~--~(\ref{eq:ns04}) by adding a source term to the right-hand side of each equation, so the modified set of equations conveniently has the analytical solution chosen \textit{a priori}.

\citet{Roy2002} introduce the general form of the two-dimensional primitive manufactured solution variables to be  a function of sines and cosines in $x$ and $y$. In this work, \citet{Roy2002}'s manufactured solutions are modified in order to address temporal accuracy as well:
\begin{equation}
 \label{eq:manufactured01}
  \phi (x,y,t) = \phi_0+ \phi_x\, f_s \left(\frac{a_{\phi x} \pi x}{L} \right) +  \phi_y \,f_s\left(\frac{a_{\phi y} \pi y}{L}\right) + \phi_t \,f_s\left(\frac{a_{\phi t} \pi t}{L}\right),
\end{equation}
where $\phi=\rho,u,v,p$ or $\sa$, and $f_s(\cdot)$ functions denote either sine or cosine function. Note that in this case, $\phi_x$, $\phi_y$  and $\phi_t$ are constants  and the subscripts do not denote differentiation.

The manufactured analytical solutions (\ref{eq:manufactured01}) for each one of the variables in two-dimensional case of FANS equations with SA turbulence model~are:
 \begin{equation}
 \begin{split}
\label{eq:manufactured_2d}
\rho\left(x,y,t\right) &=  \rho_{0}+ \rho_{x} \sin\left(\frac{a_{ \rho x} \pi x}{L}\right)+ \rho_{y} \cos\left(\frac{a_{ \rho y} \pi y}{L}\right)+ \rho_t \sin\left(\dfrac{a_{\rho t} \pi t}{L}\right),\\
u\left(x,y,t\right) &= u_{0}+u_{x} \sin\left(\frac{a_{u x} \pi x}{L}\right)+u_{y} \cos\left(\frac{a_{u y} \pi y}{L}\right) + u_t \cos\left(\dfrac{a_{u t} \pi t}{L}\right),\\
v\left(x,y,t\right) &= v_{0}+v_{x} \cos\left(\frac{a_{v x} \pi x}{L}\right)+v_{y} \sin\left(\frac{a_{v y} \pi y}{L}\right)+ v_t \sin\left(\dfrac{a_{v t} \pi t}{L}\right),\\
p\left(x,y,t\right) &= p_{0}+p_{x} \cos\left(\frac{a_{p x} \pi x}{L}\right)+p_{y} \sin\left(\frac{a_{p y} \pi y}{L}\right)+ p_t \cos\left(\dfrac{a_{p t} \pi t}{L}\right),\\
\sa(x,y,t) &= \nu_{\tsa 0} +\nu_{\tsa x} \cos\left(\frac{a_{\sa x} \pi x}{L}\right) + \nu_{\tsa y} \cos\left(\frac{a_{\sa y} \pi y}{L}\right) + \nu_{\tsa t}\cos\left(\frac{a_{\sa t} \pi t}{L}\right).
\end{split}
\end{equation}

Source terms  for mass conservation ($Q_\rho$), momentum ($Q_u$, and $Q_v$), total energy ($Q_{E}$) and SA variable ($Q_{\sa}$) equations are obtained by symbolic manipulations of FANS equations with SA turbulence model above using Maple~13~\citep{Maple} and are presented in the following sections.



 \subsection{2D FANS equations and SA turbulence model}\label{NS+SA}



MMS applied to the 2D transient FANS equations with SA turbulent model simply consists in modifying Equations~(\ref{eq:ns01}) -- (\ref{eq:ns04}) by adding a source term to the right-hand side of each equation:
% \begin{equation}
% \begin{split}
% \label{eq:ns2d_mod}
% &\Diff{\rho}{t} + \Diff{\rho u}{x}+\Diff{\rho v}{y} = Q_\rho\\
% &\Diff{\rho u}{t} +  \Diff{\rho u^2 }{x}+\Diff{\rho uv}{y} +\Diff{p}{x}-\Diff{2(\mu+\mu_t)S_{xx}}{x}-\Diff{2(\mu+\mu_t)S_{xy}}{y}= Q_u\\
% &\Diff{\rho v}{t} +  \Diff{\rho uv}{x}+\Diff{\rho v^2}{y} +\Diff{p}{y}-\Diff{2(\mu+\mu_t)S_{yx}}{x}-\Diff{2(\mu+\mu_t)S_{yy}}{y}= Q_v\\
% &  \Diff{\rho e_t}{t} +\Diff{\rho ue_t }{x}+\Diff{\rho ve_t}{y}+\Diff{pu}{x} +\Diff{pv}{y}+\Diff{q_x}{x}+\Diff{q_y}{y}-\Diff{2(\mu+\mu_t)(u S_{xx}+v S_{xy})}{x}-\Diff{2(\mu+\mu_t)(u S_{yx}+ v S_{yy})}{y}= Q_{e_t}
% \end{split}
% \end{equation}
 \begin{equation}
 \begin{split} \label{eq:ns2d_mod}
 &\Diff{\bar{\rho}}{t} +\nabla \cdot \left(\bar{\rho} \tilde{\bv{u}}\right) = Q_{\brho},\\
 &\Diff{\bar{\rho} \tilde{\bv{u}}}{t} +\nabla\cdot\left(\bar{\rho} \tilde{\bv{u}}\tilde{\bv{u}}\right) +\nabla \bar{p} -  \nabla \cdot ( 2(\bmu+\mu_t) \tilde{\bv{S}} )= Q_{\bv{\tu}},\\
 & \Diff{\bar{\rho} \tilde{E}}{t}+ \nabla \cdot (\bar{\rho} \tilde{\bv{u}} \tilde{H})-\nabla \cdot \bar{\bv{q}} - \nabla\cdot(2 (\bmu+\mu_t) \tilde{\bv{S}} \cdot \tilde{\bv{u}})= Q_{\tE},\\
 &\Diff{\bar{\rho} \sa}{t} +\nabla \cdot (\bar{\rho} \tilde{\bv{u}} \sa) - c_{b1} S_\tsa \bar{\rho} \sa - \dfrac{1 }{\sigma}\nabla \cdot \left( (\bmu+\bar{\rho}  \sa) \nabla \sa\right) -\dfrac{c_{b2} \bar{\rho} }{\sigma} \nabla \sa \cdot \nabla \sa =Q_{\sa}
 \end{split}
 \end{equation}
so the modified set of Equations (\ref{eq:ns2d_mod}) has Equation (\ref{eq:manufactured_2d}) as analytical solution. Note that in Equation (\ref{eq:ns2d_mod}) it is assumed that the distance to the neartest no slip wall is sufficiently large, i.e., $d\rightarrow\infty$.

Recall that the averaged kinematic viscosity, total energy per unit mass and the total enthalpy per unit mass are give, respectively, by:
\begin{equation}
 \begin{split}\label{eq:closure01}
  \tilde{\nu}=\dfrac{\bar{\mu}}{\bar{\rho}},\qquad \tilde{E}=\tilde{e}+\dfrac{\tilde{\bv{u}}\cdot \tilde{\bv{u}}}{2}, \quad \tilde{H}=\tilde{h}+\dfrac{\tilde{\bv{u}}\cdot \tilde{\bv{u}}}{2}
 \end{split}
\end{equation}
with $\tilde{e}$ and $\tilde{h}$ defined in Equation (\ref{eq:closure}) and $\bmu$ is the averaged absolute viscosity. The laminar mean heat-flux vector $\bar{\bv{q}}=(\bar{q}_x,\bar{q}_y)$ is given by:
%
\begin{equation}\label{eq:closure02}
 \bar{q}_x = \left(\dfrac{\bar{\mu}}{\Pr}+\dfrac{\mu_t}{\Pr_t}\right)\diff{\tilde{h}}{x}\quad \mbox{and} \quad \bar{q}_y = \left(\dfrac{\bar{\mu}}{\Pr}+\dfrac{\mu_t}{\Pr_t}\right)\diff{\tilde{h}}{y}
 \end{equation}
where the Prandtl number $\Pr$ and the turbulent Prandtl number $\Pr_t$ are assumed to be constant.

Since the fluid is considered to have constant viscosity and the distance to the neartest no slip wall is assumed to be sufficient large, i.e., $d\rightarrow\infty$, the expressions in (\ref{eq:closure}) are simplified accordingly:
\begin{gather}
\label{eq:closure03}
 \bar{\mu} = \text{constant}, \quad \bar{p} = \bar{\rho} R \tilde{T}, \quad \tilde{e} = c_v \tilde{T}, \quad \tilde{h} = c_p \tilde{T} = \tilde{e} + \frac{\bar{p}}{\bar{\rho}} \\
\label{eq:closure04}
\mu_t = \bar{\rho} \nu_t = \bar{\rho} \sa f_{v1}, \quad f_{v1} = \frac{\chi^3}{\chi^3 + c_{v1}^3}, \quad \chi = \frac{\sa}{\tilde{\nu}}, \\
\label{eq:closure05}
S_{\mathrm{sa}} = \Omega, \quad \Omega = \sqrt{\left( \diff{\tilde{u}}{y} - \diff{\tilde{v}}{x}\right)^2} \\
\label{eq:closure06}
\tS_{xx}= \diff{\tilde{u}}{x}-\dfrac{1}{3} \nabla \cdot \tilde{\bv{u}},\quad \tS_{yy}= \diff{\tilde{v}}{y}-\dfrac{1}{3} \nabla \cdot \tilde{\bv{u}}, \quad \tS_{xy}= \tS_{yx}= \left( \diff{\tilde{u}}{y} + \diff{\tilde{v}}{x}\right),
\end{gather}
with
\begin{equation}
\tilde{\bv{S}}=\left[
\begin{array}{cc}
\tS_{xx} & \tS_{xy}\\
\tS_{yx} & \tS_{yy}
\end{array}\right] .
\end{equation}

% and  variable $\tilde{\bv{S}} $ ($\tS_{ij}$ in Equation. \ref{eq:closure05}) is the instantaneous viscous stress tensor:
% \begin{equation}\label{eq:closure06}
% \tS_{xx}= \diff{\tilde{u}}{x}-\dfrac{1}{3} \nabla \cdot \tilde{\bv{u}},\quad
% \tS_{yy}= \diff{\tilde{v}}{y}-\dfrac{1}{3} \nabla \cdot \tilde{\bv{u}}, \quad
% \tS_{xy}= \tS_{yx}= \left( \diff{\tilde{u}}{y} + \diff{\tilde{v}}{x}\right),
% \end{equation}



Source terms $Q_{\brho}$, $Q_{\tu}$, $Q_{\tv}$, $Q_{\tE}$ and $Q_{\sa}$ are presented in the subsequent sessions with the use of the auxiliary variables:
\begin{equation}
 \begin{split}
\label{eq:aux_2d}
\Rho &= \rho_{0}+ \rho_{x} \sin\left(\frac{a_{ \rho x} \pi x}{L}\right)+ \rho_{y} \cos\left(\frac{a_{ \rho y} \pi y}{L}\right)+ \rho_t \sin\left(\dfrac{a_{\rho t} \pi t}{L}\right),\\
\U &= u_{0}+u_{x} \sin\left(\frac{a_{u x} \pi x}{L}\right)+u_{y} \cos\left(\frac{a_{u y} \pi y}{L}\right) + u_t \cos\left(\dfrac{a_{u t} \pi t}{L}\right),\\
\V &= v_{0}+v_{x} \cos\left(\frac{a_{v x} \pi x}{L}\right)+v_{y} \sin\left(\frac{a_{v y} \pi y}{L}\right)+ v_t \sin\left(\dfrac{a_{v t} \pi t}{L}\right),\\
\PP &= p_{0}+p_{x} \cos\left(\frac{a_{p x} \pi x}{L}\right)+p_{y} \sin\left(\frac{a_{p y} \pi y}{L}\right)+ p_t \cos\left(\dfrac{a_{p t} \pi t}{L}\right),\\
\Nu &= \nu_{\tsa 0} +\nu_{\tsa x} \cos\left(\frac{a_{\sa x} \pi x}{L}\right) + \nu_{\tsa y} \cos\left(\frac{a_{\sa y} \pi y}{L}\right) + \nu_{\tsa t}\cos\left(\frac{a_{\sa t} \pi t}{L}\right).
\end{split}
\end{equation}


\subsubsection{2D FANS Mass Conservation}

The 2D mass conservation equation written as an operator is:
\begin{equation*}
  \Lo=\Diff{\brho }{t} +  \Diff{\brho \tu}{x}+\Diff{\brho \tv}{y}.
\end{equation*}

Analytically differentiating Equation (\ref{eq:manufactured_2d}) for $\brho$, $\tu$ and $\tv$ using operator $\Lo$ defined above gives  the source term~$Q_{\brho}$:
\begin{equation}
 \begin{split}
Q_{\brho} &= \dfrac{a_{\rho x} \pi \rho_x \U }{L}\cos\left(\dfrac{a_{\rho x} \pi x}{L}\right)+\\
&-\dfrac{a_{\rho y} \pi \rho_y \V }{L}\sin\left(\dfrac{a_{\rho y} \pi y}{L}\right)+\\
&+\dfrac{\pi \Rho}{L}\left[a_{ux} u_x \cos\left(\dfrac{a_{ux} \pi x}{L}\right)+a_{vy} v_y \cos\left(\dfrac{a_{vy} \pi y}{L}\right)\right] +\\
&+\dfrac{a_{\rho t} \pi \rho_t }{L}\cos\left(\dfrac{a_{\rho t} \pi t}{L}\right).
 \end{split}
\end{equation}
where $\Rho,\,\U$ and $\V$ are given in Equation (\ref{eq:aux_2d}).

\subsubsection{2D FANS Momentum Conservation}

For the generation of the analytical source term $Q_{\tu}$ for the $x$-momentum equation, the first component of Equation~(\ref{eq:ns02}) is written as an  operator $\Lo$:
\begin{equation*}
 \Lo= \Diff{\brho \tu}{t} +\Diff{\brho \tu^2 }{x}+\Diff{\brho \tu\tv}{y} +\Diff{\bp}{x}-\Diff{2(\bmu+\mu_t)\tS_{xx}}{x}-\Diff{2(\bmu+\mu_t)\tS_{xy}}{y},
\end{equation*}
which, when operated in Equation (\ref{eq:manufactured_2d}), provides source term $Q_{\tu}$:
\begin{equation}
 \begin{split}
 Q_{\tu} &= \dfrac{a_{\rho x} \pi \rho_x \U^2 }{L} \cos\left(\dfrac{a_{\rho x} \pi x}{L}\right)+ \\
&-\dfrac{a_{\rho y} \pi \rho_y \U \V  }{L}\sin\left(\dfrac{a_{\rho y} \pi y}{L}\right)+ \\
&-\dfrac{a_{uy} \pi u_y \Rho \V  }{L}\sin\left(\dfrac{a_{uy} \pi y}{L}\right)+ \\
&+ \dfrac{\pi \Rho \U}{L}\left[2 a_{ux} u_x  \cos\left(\dfrac{a_{ux} \pi x}{L}\right)+a_{vy} v_y  \cos\left(\dfrac{a_{vy} \pi y}{L}\right)\right]+ \\
&-\dfrac{a_{px} \pi p_x  }{L}\sin\left(\dfrac{a_{px} \pi x}{L}\right)+ \\
&+\dfrac{f_{v1} \pi^2 \Rho \Nu}{L^2}\left[\nicefrac{4}{3} \; a_{ux}^2 u_x  \sin\left(\dfrac{a_{ux} \pi x}{L}\right)+a_{uy}^2 u_y  \cos\left(\dfrac{a_{uy} \pi y}{L}\right)\right] + \\
&+\dfrac{f_{v1} \pi^2 \Rho}{L^2}\left[\nicefrac{4}{3} \; a_{ux} a_{\nu_{sa} x} u_x \nu_{sa x}  \cos\left(\dfrac{a_{ux} \pi x}{L}\right)  \sin\left(\dfrac{a_{\nu_{sa} x} \pi x}{L}\right)- a_{uy} a_{\nu_{sa} y} u_y \nu_{sa y} \sin\left(\dfrac{a_{uy} \pi y}{L}\right)  \sin\left(\dfrac{a_{\nu_{sa} y} \pi y}{L}\right)\right.+\\
    &\qquad\left.- a_{vx} a_{\nu_{sa} y} v_x \nu_{sa y} \sin\left(\dfrac{a_{vx} \pi x}{L}\right)  \sin\left(\dfrac{a_{\nu_{sa} y} \pi y}{L}\right)-\nicefrac{2}{3} \;  a_{vy} a_{\nu_{sa} x} v_y \nu_{sa x} \cos\left(\dfrac{a_{vy} \pi y}{L}\right)  \sin\left(\dfrac{a_{\nu_{sa} x} \pi x}{L}\right)\right] + \\
&+\dfrac{f_{v1} \pi^2 \Nu}{L^2}\left[-\nicefrac{4}{3} \; a_{\rho x} a_{ux} \rho_x u_x  \cos\left(\dfrac{a_{\rho x} \pi x}{L}\right)  \cos\left(\dfrac{a_{ux} \pi x}{L}\right)+\nicefrac{2}{3} \; a_{\rho x} a_{vy} \rho_x v_y  \cos\left(\dfrac{a_{\rho x} \pi x}{L}\right)  \cos\left(\dfrac{a_{vy} \pi y}{L}\right)\right.+\\
    &\qquad\left.-a_{\rho y} a_{uy} \rho_y u_y  \sin\left(\dfrac{a_{\rho y} \pi y}{L}\right)  \sin\left(\dfrac{a_{uy} \pi y}{L}\right)-a_{\rho y} a_{vx} \rho_y v_x  \sin\left(\dfrac{a_{\rho y} \pi y}{L}\right)  \sin\left(\dfrac{a_{vx} \pi x}{L}\right)\right] + \\
&+\dfrac{\pi^2 \bmu}{3L^2}\left[\dfrac{c_{v1}^3 }{ \chi^3+c_{v1}^3} +f_{v1}\right]\left[4 a_{ux}^2 u_x  \sin\left(\dfrac{a_{ux} \pi x}{L}\right)+3 a_{uy}^2 u_y  \cos\left(\dfrac{a_{uy} \pi y}{L}\right)\right] +\\
&+\dfrac{a_{\rho t} \pi \rho_t \U  }{L}\cos\left(\dfrac{a_{\rho t} \pi t}{L}\right)+ \\
&-\dfrac{a_{ut} \pi u_t \Rho  }{L}\sin\left(\dfrac{a_{ut} \pi t}{L}\right),
 \end{split}
\end{equation}
where
\begin{equation}\label{eq:chi}
  f_{v1} = \dfrac{\chi^3}{\chi^3+c_{v1}^3}\quad \mbox{and}\quad \chi =\dfrac{\Nu}{\tilde{\nu}} =\dfrac{\Rho \Nu}{\bmu},
\end{equation}
and $\Rho,\,\U,\,\V$ and $\Nu$ are given in Equation (\ref{eq:aux_2d}).


Analogously, for the generation of the analytical source term $Q_{\tv}$ for the $y$-momentum equation, the second component of Equation  (\ref{eq:ns02})  is written as an  operator $\Lo$:
\begin{equation*}
 \Lo =\Diff{\brho \tv}{t} +\Diff{\brho \tu\tv}{x}+\Diff{\brho \tv^2}{y} +\Diff{\bp}{y}-\Diff{2(\bmu+\mu_t)\tS_{yx}}{x}-\Diff{2(\bmu+\mu_t)\tS_{yy}}{y},
\end{equation*}
and then applied to Equation  (\ref{eq:manufactured_2d}). It yields:

\begin{equation}
 \begin{split}
Q_{\tv} &= \dfrac{a_{\rho x} \pi \rho_x \U \V  }{L}\cos\left(\dfrac{a_{\rho x} \pi x}{L}\right)+ \\
&-\dfrac{a_{\rho y} \pi \rho_y \V^2 }{L} \sin\left(\dfrac{a_{\rho y} \pi y}{L}\right)+ \\
&-\dfrac{a_{vx} \pi v_x \Rho \U  }{L}\sin\left(\dfrac{a_{vx} \pi x}{L}\right)+ \\
&+\dfrac{ \pi \Rho \V}{L}\left[a_{ux} u_x  \cos\left(\dfrac{a_{ux} \pi x}{L}\right)+2 a_{vy} v_y  \cos\left(\dfrac{a_{vy} \pi y}{L}\right)\right]+ \\
&+\dfrac{a_{py} \pi p_y  }{L}\cos\left(\dfrac{a_{py} \pi y}{L}\right)+ \\
&+\dfrac{f_{v1} \pi^2 \Rho \Nu}{L^2}\left[a_{vx}^2 v_x  \cos\left(\dfrac{a_{vx} \pi x}{L}\right)+\nicefrac{4}{3} \; a_{vy}^2 v_y  \sin\left(\dfrac{a_{vy} \pi y}{L}\right)\right] + \\
&+\dfrac{f_{v1} \pi^2 \Rho}{L^2}\left[-\nicefrac{2}{3} \; a_{ux} a_{\nu_{sa} y} u_x \nu_{sa y}  \cos\left(\dfrac{a_{ux} \pi x}{L}\right)  \sin\left(\dfrac{a_{\nu_{sa} y} \pi y}{L}\right)- a_{uy} a_{\nu_{sa} x} u_y \nu_{sa x} \sin\left(\dfrac{a_{uy} \pi y}{L}\right)  \sin\left(\dfrac{a_{\nu_{sa} x} \pi x}{L}\right)\right.+\\
    &\qquad -\left. a_{vx} a_{\nu_{sa} x} v_x \nu_{sa x} \sin\left(\dfrac{a_{vx} \pi x}{L}\right)  \sin\left(\dfrac{a_{\nu_{sa} x} \pi x}{L}\right)+\nicefrac{4}{3} \;  a_{vy} a_{\nu_{sa} y} v_y \nu_{sa y} \cos\left(\dfrac{a_{vy} \pi y}{L}\right)  \sin\left(\dfrac{a_{\nu_{sa} y} \pi y}{L}\right)\right] + \\
&+\dfrac{f_{v1} \pi^2 \Nu }{L^2}\left[a_{\rho x} a_{uy} \rho_x u_y  \cos\left(\dfrac{a_{\rho x} \pi x}{L}\right)  \sin\left(\dfrac{a_{uy} \pi y}{L}\right)+a_{\rho x} a_{vx} \rho_x v_x  \cos\left(\dfrac{a_{\rho x} \pi x}{L}\right)  \sin\left(\dfrac{a_{vx} \pi x}{L}\right)\right.+\\
    &\qquad\left. -\nicefrac{2}{3} \; a_{\rho y} a_{ux} \rho_y u_x  \sin\left(\dfrac{a_{\rho y} \pi y}{L}\right)  \cos\left(\dfrac{a_{ux} \pi x}{L}\right)+\nicefrac{4}{3} \; a_{\rho y} a_{vy} \rho_y v_y  \sin\left(\dfrac{a_{\rho y} \pi y}{L}\right)  \cos\left(\dfrac{a_{vy} \pi y}{L}\right)\right] + \\
&+ \dfrac{\pi^2 \bmu}{3L^2}\left[\dfrac{c_{v1}^3 }{ \chi^3+c_{v1}^3} +f_{v1}\right]\left[3 a_{vx}^2 v_x  \cos\left(\dfrac{a_{vx} \pi x}{L}\right)+4 a_{vy}^2 v_y  \sin\left(\dfrac{a_{vy} \pi y}{L}\right)\right] +\\
&+\dfrac{a_{\rho t} \pi \rho_t \V}{L}  \cos\left(\dfrac{a_{\rho t} \pi t}{L}\right)+ \\
&+\dfrac{a_{vt} \pi v_t \Rho }{L} \cos\left(\dfrac{a_{vt} \pi t}{L}\right),
 \end{split}
\end{equation}
where $\chi$ and $f_{v1}$ are given in  Equation (\ref{eq:chi}), and  $\Rho,\,\U,\,\V$ and $\Nu$ are given in Equation (\ref{eq:aux_2d}).


\subsubsection{2D FANS Total Energy Conservation}

The operator for the 2D Navier--Stokes total energy is:
\begin{equation*}
\begin{split}
\Lo&=  \Diff{\brho \tE }{t}+ \Diff{\brho \tu \tE }{x}+\Diff{\brho \tv \tE}{y}+\Diff{\bp\tu}{x} +\Diff{\bp\tv}{y} + \Diff{\bq_x}{x} +\Diff{\bq_y}{y} +\\
  &-\Diff{2(\bmu+\mu_t)\tu \tS_{xx}}{x}-\Diff{2(\bmu+\mu_t)\tv \tS_{xy}}{x}-\Diff{2(\bmu+\mu_t) \tu \tS_{yx}}{y}-\Diff{2(\bmu+\mu_t)\tv \tS_{yy}}{y}.
\end{split}
\end{equation*}

Source term $Q_{\tE}$ is obtained by operating $\Lo$ on Equation  (\ref{eq:manufactured_2d}) together with the use of the  auxiliary relations for energy given in Equations (\ref{eq:closure03}), (\ref{eq:closure04}) and (\ref{eq:closure06}). It yields:

\begin{equation*}
 \begin{split}
Q_{\tE}&= \dfrac{ a_{\rho x} \pi \rho_x \U(\U^2+\V^2) }{2L}\cos\left(\dfrac{a_{\rho x} \pi x}{L}\right)\quad -\quad \dfrac{ a_{\rho y} \pi \rho_y \V(\U^2+ \V^2) }{2L}\sin\left(\dfrac{a_{\rho y} \pi y}{L}\right)\quad +\quad \dfrac{a_{\rho t} \pi \rho_t (\U^2+\V^2) }{2L}\cos\left(\dfrac{a_{\rho t} \pi t}{L}\right)+\\
&-\dfrac{c_p a_{px} \pi p_x \U }{LR}\sin\left(\dfrac{a_{px} \pi x}{L}\right)\quad+\quad\dfrac{c_p a_{py} \pi p_y \V }{LR}\cos\left(\dfrac{a_{py} \pi y}{L}\right)+\\
&+\dfrac{c_p \pi \PP}{LR}\left[a_{ux} u_x \cos\left(\dfrac{a_{ux} \pi x}{L}\right)+a_{vy} v_y \cos\left(\dfrac{a_{vy} \pi y}{L}\right)\right] +\\
&-\dfrac{a_{ut} \pi u_t \Rho \U }{L}\sin\left(\dfrac{a_{ut} \pi t}{L}\right)\quad+\quad\dfrac{a_{vt} \pi v_t \Rho \V }{L}\cos\left(\dfrac{a_{vt} \pi t}{L}\right)+\\
&+ \dfrac{\pi^2  \U \mu_t}{L^2}\left[\nicefrac{4}{3}\, a_{ux}^2 u_x \sin\left(\dfrac{a_{ux} \pi x}{L}\right)+a_{uy}^2 u_y \cos\left(\dfrac{a_{uy} \pi y}{L}\right)\right]+\\
&+\dfrac{\pi^2  \V \mu_t}{L^2}\left[a_{vx}^2 v_x \cos\left(\dfrac{a_{vx} \pi x}{L}\right)+\nicefrac{4}{3}\, a_{vy}^2 v_y \sin\left(\dfrac{a_{vy} \pi y}{L}\right)\right] +\\
&+\dfrac{\pi^2 \mu_t}{L^2}\left[-\nicefrac{4}{3}\, a_{ux}^2 u_x^2 \cos\left(\dfrac{a_{ux} \pi x}{L}\right)^2+\nicefrac{4}{3}\, a_{ux} a_{vy} u_x v_y \cos\left(\dfrac{a_{ux} \pi x}{L}\right) \cos\left(\dfrac{a_{vy} \pi y}{L}\right)-a_{uy}^2 u_y^2 \sin\left(\dfrac{a_{uy} \pi y}{L}\right)^2\right.+\\
   &\qquad\left.-2 a_{uy} a_{vx} u_y v_x \sin\left(\dfrac{a_{uy} \pi y}{L}\right) \sin\left(\dfrac{a_{vx} \pi x}{L}\right)-a_{vx}^2 v_x^2 \sin\left(\dfrac{a_{vx} \pi x}{L}\right)^2-\nicefrac{4}{3}\, a_{vy}^2 v_y^2 \cos\left(\dfrac{a_{vy} \pi y}{L}\right)^2\right] +\\
&+\dfrac{\pi^2 f_{v1} \Rho \U}{L^2}\left[\nicefrac{4}{3}\, a_{ux} a_{\sa x} u_x \nu_{sa x} \cos\left(\dfrac{a_{ux} \pi x}{L}\right) \sin\left(\dfrac{a_{\sa x} \pi x}{L}\right)-a_{uy} a_{\sa y} u_y \nu_{sa y} \sin\left(\dfrac{a_{uy} \pi y}{L}\right) \sin\left(\dfrac{a_{\sa y} \pi y}{L}\right)\right.+\\
     &\qquad\left.-a_{vx} a_{\sa y} v_x \nu_{sa y} \sin\left(\dfrac{a_{vx} \pi x}{L}\right) \sin\left(\dfrac{a_{\sa y} \pi y}{L}\right)-\nicefrac{2}{3}\, a_{vy} a_{\sa x} v_y \nu_{sa x} \cos\left(\dfrac{a_{vy} \pi y}{L}\right) \sin\left(\dfrac{a_{\sa x} \pi x}{L}\right)\right] +\\
&+\dfrac{ \pi^2 f_{v1} \Rho \V}{L^2}\left[-\nicefrac{2}{3}\, a_{ux} a_{\sa y} u_x \nu_{sa y} \cos\left(\dfrac{a_{ux} \pi x}{L}\right) \sin\left(\dfrac{a_{\sa y} \pi y}{L}\right)-a_{uy} a_{\sa x} u_y \nu_{sa x} \sin\left(\dfrac{a_{uy} \pi y}{L}\right) \sin\left(\dfrac{a_{\sa x} \pi x}{L}\right)\right.+\\
    &\qquad\left.-a_{vx} a_{\sa x} v_x \nu_{sa x} \sin\left(\dfrac{a_{vx} \pi x}{L}\right) \sin\left(\dfrac{a_{\sa x} \pi x}{L}\right)+\nicefrac{4}{3}\, a_{vy} a_{\sa y} v_y \nu_{sa y} \cos\left(\dfrac{a_{vy} \pi y}{L}\right) \sin\left(\dfrac{a_{\sa y} \pi y}{L}\right)\right]+\\
&+\dfrac{\pi^2 f_{v1} \U \Nu}{L^2}\left[-\nicefrac{4}{3}\, a_{\rho x} a_{ux} \rho_x u_x \cos\left(\dfrac{a_{\rho x} \pi x}{L}\right) \cos\left(\dfrac{a_{ux} \pi x}{L}\right)+\nicefrac{2}{3}\, a_{\rho x} a_{vy} \rho_x v_y \cos\left(\dfrac{a_{\rho x} \pi x}{L}\right) \cos\left(\dfrac{a_{vy} \pi y}{L}\right)\right.+\\
    &\qquad\left.-a_{\rho y} a_{uy} \rho_y u_y \sin\left(\dfrac{a_{\rho y} \pi y}{L}\right) \sin\left(\dfrac{a_{uy} \pi y}{L}\right)-a_{\rho y} a_{vx} \rho_y v_x \sin\left(\dfrac{a_{\rho y} \pi y}{L}\right) \sin\left(\dfrac{a_{vx} \pi x}{L}\right)\right] +\\
&+\dfrac{\pi^2 f_{v1} \V \Nu}{L^2}\left[a_{\rho x} a_{uy} \rho_x u_y \cos\left(\dfrac{a_{\rho x} \pi x}{L}\right) \sin\left(\dfrac{a_{uy} \pi y}{L}\right)+a_{\rho x} a_{vx} \rho_x v_x \cos\left(\dfrac{a_{\rho x} \pi x}{L}\right) \sin\left(\dfrac{a_{vx} \pi x}{L}\right)\right.+\\
    &\qquad\left.-\nicefrac{2}{3}\, a_{\rho y} a_{ux} \rho_y u_x \sin\left(\dfrac{a_{\rho y} \pi y}{L}\right) \cos\left(\dfrac{a_{ux} \pi x}{L}\right)+\nicefrac{4}{3}\, a_{\rho y} a_{vy} \rho_y v_y \sin\left(\dfrac{a_{\rho y} \pi y}{L}\right) \cos\left(\dfrac{a_{vy} \pi y}{L}\right)\right] +\\
&+\dfrac{c_v a_{\rho t} \pi \rho_t \PP }{L R \Rho}\cos\left(\dfrac{a_{\rho t} \pi t}{L}\right) +\\
&-\dfrac{\pi^2 \bmu}{3L^2}\left[\dfrac{c_{v1}^3 }{\chi^3+c_{v1}^3}+f_{v1}\right] \left[4 a_{ux}^2 u_x^2 \cos\left(\dfrac{a_{ux} \pi x}{L}\right)^2-4 a_{ux} a_{vy} u_x v_y \cos\left(\dfrac{a_{ux} \pi x}{L}\right) \cos\left(\dfrac{a_{vy} \pi y}{L}\right)+3 a_{uy}^2 u_y^2 \sin\left(\dfrac{a_{uy} \pi y}{L}\right)^2\right.+\\
    &\qquad\left.+6 a_{uy} a_{vx} u_y v_x \sin\left(\dfrac{a_{uy} \pi y}{L}\right) \sin\left(\dfrac{a_{vx} \pi x}{L}\right)+3 a_{vx}^2 v_x^2 \sin\left(\dfrac{a_{vx} \pi x}{L}\right)^2+4 a_{vy}^2 v_y^2 \cos\left(\dfrac{a_{vy} \pi y}{L}\right)^2\right] +\\
&+ \dfrac{ \pi^2 \bmu \U}{3L^2}\left[\dfrac{c_{v1}^3 }{ \chi^3+c_{v1}^3} +f_{v1}\right] \left[4 a_{ux}^2 u_x \sin\left(\dfrac{a_{ux} \pi x}{L}\right)+3 a_{uy}^2 u_y \cos\left(\dfrac{a_{uy} \pi y}{L}\right)\right]+\\
&+ \dfrac{\pi^2 \bmu \V}{3L^2}\left[\dfrac{c_{v1}^3 }{ \chi^3+c_{v1}^3} +f_{v1}\right] \left[3 a_{vx}^2 v_x \cos\left(\dfrac{a_{vx} \pi x}{L}\right)+4 a_{vy}^2 v_y \sin\left(\dfrac{a_{vy} \pi y}{L}\right)\right] +\\
&+\dfrac{\pi \Rho \U^2}{2L} \left[3 a_{ux} u_x \cos\left(\dfrac{a_{ux} \pi x}{L}\right)+a_{vy} v_y \cos\left(\dfrac{a_{vy} \pi y}{L}\right)\right] +\\
&-\dfrac{\pi \Rho \U \V}{L}\left[a_{uy} u_y \sin\left(\dfrac{a_{uy} \pi y}{L}\right)+a_{vx} v_x \sin\left(\dfrac{a_{vx} \pi x}{L}\right)\right] +\\
&+\dfrac{ \pi \Rho \V^2}{2L} \left[a_{ux} u_x \cos\left(\dfrac{a_{ux} \pi x}{L}\right)+3 a_{vy} v_y \cos\left(\dfrac{a_{vy} \pi y}{L}\right)\right]+\\
&+\cdots
 \end{split}
\end{equation*}
\begin{equation}\label{eq:ns_2d_e}
\begin{split}
&- \dfrac{c_p \pi^2}{L^2 R \Rho \Nu}\dfrac{\mu_t}{\Pr_t}\left[a_{px} a_{\sa x} p_x \nu_{sa x} \sin\left(\dfrac{a_{px} \pi x}{L}\right) \sin\left(\dfrac{a_{\sa x} \pi x}{L}\right)-a_{py} a_{\sa y} p_y \nu_{sa y} \cos\left(\dfrac{a_{py} \pi y}{L}\right) \sin\left(\dfrac{a_{\sa y} \pi y}{L}\right)\right]+\\
 &- \dfrac{c_p \pi^2 \PP}{L^2 R \Rho^2 \Nu}\dfrac{\mu_t}{\Pr_t} \left[a_{\rho x} a_{\sa x} \rho_x \nu_{sa x} \cos\left(\dfrac{a_{\rho x} \pi x}{L}\right) \sin\left(\dfrac{a_{\sa x} \pi x}{L}\right)-a_{\rho y} a_{\sa y} \rho_y \nu_{sa y} \sin\left(\dfrac{a_{\rho y} \pi y}{L}\right) \sin\left(\dfrac{a_{\sa y} \pi y}{L}\right)\right] +\\
 &-\dfrac{c_p \pi^2}{ L^2 R \Rho^2}\left[\dfrac{\mu_t}{\Pr_t}+\dfrac{2\bmu}{\Pr}\right]\left[a_{px} a_{\rho x} \rho_x p_x \cos\left(\dfrac{a_{\rho x} \pi x}{L}\right) \sin\left(\dfrac{a_{px} \pi x}{L}\right)+a_{py} a_{\rho y} \rho_y p_y \sin\left(\dfrac{a_{\rho y} \pi y}{L}\right) \cos\left(\dfrac{a_{py} \pi y}{L}\right)\right] +\\
 &-\dfrac{c_p \pi^2 \PP}{ L^2 R \Rho^3}\left[\dfrac{\mu_t}{\Pr_t}+\dfrac{2\bmu}{\Pr}\right]\left[a_{\rho x}^2 \rho_x^2 \cos\left(\dfrac{a_{\rho x} \pi x}{L}\right)^2+a_{\rho y}^2 \rho_y^2 \sin\left(\dfrac{a_{\rho y} \pi y}{L}\right)^2\right] +\\
 &+\dfrac{c_p \pi^2}{L^2 R \Rho}\left[\dfrac{\mu_t}{\Pr_t}+\dfrac{\bmu}{\Pr}\right] \left[a_{px}^2 p_x \cos\left(\dfrac{a_{px} \pi x}{L}\right)+a_{py}^2 p_y \sin\left(\dfrac{a_{py} \pi y}{L}\right)\right] +\\
%
&-\dfrac{c_p \pi^2 \PP}{L^2 R \Rho^2}\left[\dfrac{\mu_t}{\Pr_t}+\dfrac{\bmu}{\Pr}\right] \left[a_{\rho x}^2 \rho_x \sin\left(\dfrac{a_{\rho x} \pi x}{L}\right)+a_{\rho y}^2 \rho_y \cos\left(\dfrac{a_{\rho y} \pi y}{L}\right)\right],
 \end{split}
\end{equation}
where  $\Rho,\,\PP,\, \U,\,\V$ and $\Nu$ are given in Equation (\ref{eq:aux_2d}),  $\chi$ and $f_{v1}$ are given in  Equation (\ref{eq:chi}),   and  $$\mu_t= f_{v1} \Rho \Nu.$$


\subsubsection{2D SA Transport Equation}

The operator for the viscosity-like baseline compressible Spalart-Allmaras equation is:
\begin{equation*}
\begin{split}
\Lo&=  \Diff{\brho \sa }{t}+ \Diff{\brho \tu \sa }{x}+\Diff{\brho \tv \sa}{y}- c_{b1} S_\tsa \bar{\rho} \sa +\\
   &+ \dfrac{1 }{\sigma}\left[\diff{}{x}\left((\bmu+\bar{\rho} \sa) \diff{\sa}{x}\right)+\diff{}{y}\left((\bmu+\bar{\rho} \sa) \diff{\sa}{y}\right)\right] +\dfrac{c_{b2} \bar{\rho} }{\sigma} \left[ \left(\diff{\sa}{x}\right)^2 + \left(\diff{\sa}{y}\right)^2\right]
\end{split}
\end{equation*}

Source term $Q_{\sa}$ is obtained by operating $\Lo$ on Equation  (\ref{eq:manufactured_2d}) together with the use of the  auxiliary relations for energy given in Equations (\ref{eq:closure03}) -- (\ref{eq:closure05}). It yields:

\begin{equation}
 \begin{split}
  Q_{\sa} &= \dfrac{a_{\rho x} \pi \rho_x \U \Nu  }{L}\cos\left(\dfrac{a_{\rho x} \pi x}{L}\right)+ \\
&-\dfrac{a_{\rho y} \pi \rho_y \V \Nu }{L}\sin\left(\dfrac{a_{\rho y} \pi y}{L}\right)+ \\
&-\dfrac{a_{\nu_{sa} x} \pi \nu_{sa x} \Rho \U }{L}\sin\left(\dfrac{a_{\nu_{sa} x} \pi x}{L}\right)+ \\
&-\dfrac{a_{\nu_{sa} y} \pi \nu_{sa y} \Rho \V }{L}\sin\left(\dfrac{a_{\nu_{sa} y} \pi y}{L}\right)+ \\
&-\dfrac{(1+c_{b2}) \pi^2 \Rho}{L^2 \sigma}\left[a_{\nu_{sa} x}^2 \nu_{sa x}^2  \sin\left(\dfrac{a_{\nu_{sa} x} \pi x}{L}\right)^2+a_{\nu_{sa} y}^2 \nu_{sa y}^2  \sin\left(\dfrac{a_{\nu_{sa} y} \pi y}{L}\right)^2\right]+\\
&+\dfrac{\pi^2 \Nu}{ L^2\sigma}\left[a_{\rho x} a_{\nu_{sa} x} \rho_x \nu_{sa x}  \cos\left(\dfrac{a_{\rho x} \pi x}{L}\right)  \sin\left(\dfrac{a_{\nu_{sa} x} \pi x}{L}\right)-a_{\rho y} a_{\nu_{sa} y} \rho_y \nu_{sa y}  \sin\left(\dfrac{a_{\rho y} \pi y}{L}\right)  \sin\left(\dfrac{a_{\nu_{sa} y} \pi y}{L}\right)\right] +\\
 &-c_{b1}  \pi \Rho \Nu \sqrt{\dfrac{1}{L^2} \left[-a_{uy} u_y  \sin\left(\dfrac{a_{uy} \pi y}{L}\right)+a_{vx} v_x  \sin\left(\dfrac{a_{vx} \pi x}{L}\right)\right]^2}+\\
&+\dfrac{\pi \Rho \Nu}{L}\left[a_{ux} u_x  \cos\left(\dfrac{a_{ux} \pi x}{L}\right)+a_{vy} v_y  \cos\left(\dfrac{a_{vy} \pi y}{L}\right)\right] + \\
 &+\dfrac{(\Rho \Nu+\bmu) \pi^2}{L^2 \sigma}\left[a_{\nu_{sa} x}^2 \nu_{sa x}  \cos\left(\dfrac{a_{\nu_{sa} x} \pi x}{L}\right)+a_{\nu_{sa} y}^2 \nu_{sa y}  \cos\left(\dfrac{a_{\nu_{sa} y} \pi y}{L}\right)\right]+\\
&+ \dfrac{a_{\rho t} \pi \rho_t \Nu}{L} \cos\left(\dfrac{a_{\rho t} \pi t}{L}\right)+\\
&- \dfrac{a_{\nu_{sa}t} \pi \nu_{sa_t}}{L} \Rho \sin\left(\dfrac{a_{\nu_{sa} t} \pi t}{L}\right),
 \end{split}
\end{equation}
with  $\Rho,\, \U,\,\V$ and $\Nu$ defined in (\ref{eq:aux_2d}).


\section{Comments}
The complexity, and consequently length, of the source terms increase with both dimension and physics handled by the governing equations. In some cases, such as the 2D energy equation, the final expression for $Q_{\tE}$ may reach 107,600 characters, including parenthesis and mathematical operators, prior to factorization.

Applying commands in order to simplify such extensive expression is challenging even with a very good machine; thus, a suitable alternative to this issue is to simplify the equation by dividing it into a combination of sub-operators handling different physical phenomena. Then, each one of the operators may be applied to the manufactured solutions individually, and the resulting sub-source terms are combined back to represent the source term for the original equation.



For instance, instead of writing the two-dimensional FANS energy equation using one single operator~$\Lo$:
 \begin{equation} \label{eq:ns2d05}
\Lo= \Diff{\bar{\rho} \tilde{E}}{t}+ \nabla \cdot (\bar{\rho} \tilde{\bv{u}} \tilde{H})-\nabla \cdot \bar{\bv{q}} - \nabla\cdot(2 (\bmu+\mu_t) \tilde{\bv{S}} \cdot \tilde{\bv{u}})
\end{equation}
to then be used in the MMS, let Equation (\ref{eq:ns2d05}) be written with four operators, according to their physical meaning:
\begin{equation}
 \begin{split}\label{sub01}
  \Lo_1&=\Diff{\bar{\rho} \tilde{E}}{t},\\
  \Lo_2&=\nabla \cdot (\bar{\rho} \tilde{\bv{u}} \tilde{H}),\\
  \Lo_3&=-\nabla \cdot \bar{\bv{q}},\\
  \Lo_4&=-\nabla\cdot(2 (\bmu+\mu_t) \tilde{\bv{S}} \cdot \tilde{\bv{u}}),
 \end{split}
\end{equation}
where $\Lo_1$ denotes the rate of accumulation of inertial and kinetic energy, $\Lo_2$ is the net rate of internal and kinetic energy increase by convection together with the work done on the fluid by external body forces, $\Lo_3$ is the net rate of heat addition due to heat conduction, and $\Lo_4$ is the rate of work done on the fluid by viscous forces. Naturally:
$$\Lo=\Lo_1+\Lo_2+\Lo_3+\Lo_4.$$


In fact, due to the extremely high complexity of operator $\Lo_3$ (see Eq. (\ref{eq:closure02}) and (\ref{eq:closure04})), further subdivision had to be carried out in order to allow the algebraic manipulations:
\begin{equation}
 \begin{split}\label{sub02}
 \Lo_3&=\Lo_{3a}\cdot(\Lo_{3b}+\Lo_{3c})+\Lo_{3d}\cdot\Lo_{3e}+\Lo_{3f}\cdot\Lo_{3g}\\
 \Lo_{3a} &= -\left(\dfrac{\bmu}{\Pr}+\dfrac{\mu_t}{\Pr_t}\right),\quad \Lo_{3b} = \diff{^2\hh}{ x^2},\quad \Lo_{3c} = \diff{^2\hh}{ y^2}\\
 \Lo_{3d} &= \diff{\mu_t}{ x},\quad \Lo_{3e} = -\dfrac{1}{\Pr_t} \diff{\hh}{ x},\\
 \Lo_{3f} &= \diff{\mu_t}{ y},\quad \Lo_{3g} = -\dfrac{1}{\Pr_t} \diff{\hh}{ y}.\\
 \end{split}
\end{equation}

 After the application of each sub-operator defined in (\ref{sub01}) and (\ref{sub02}), the corresponding sub-source terms are also simplified, factorized and sorted. Then, the final factorized version is checked against the original one, in order to assure that not human error has been introduced.  This strategy allowed the original  107,600 character-long  expression for $Q_{\tE}$ to be reduced to less than 12,400, and expressed in Equation (\ref{eq:ns_2d_e}).


\subsection{Boundary Conditions}
Additionally to verifying code capability of solving the governing equations accurately in the interior of the domain of interest, one may also verify the software capability of correctly imposing boundary conditions. Therefore, the gradients of the  analytical solutions (\ref{eq:manufactured_2d}) have been calculated and translated into $C$ codes. They are:
\begin{equation*}
\nabla  \brho= \left[ \begin{array}{c}
 \dfrac{  a_{\rho x}  \pi \rho_x}{L} \cos\left( \dfrac{ a_{\rho x}  \pi  x}{L}\right)\vspace{5pt} \\
-\dfrac{  a_{\rho y}  \pi \rho_y}{L} \sin\left( \dfrac{ a_{\rho y}  \pi  y}{L}\right)
\end{array} \right],
\qquad
\nabla \bp = \left[ \begin{array}{c}
- \dfrac{  a_{px}  \pi p_x}{L} \sin\left( \dfrac{ a_{px}  \pi  x}{L}\right)\vspace{5pt}\\
  \dfrac{  a_{py}  \pi p_y}{L} \cos\left( \dfrac{ a_{py}  \pi  y}{L}\right)
\end{array} \right],
\quad
\nabla \tu = \left[ \begin{array}{c}
  \dfrac{  a_{ux}  \pi u_x}{L} \cos\left( \dfrac{ a_{ux}  \pi  x}{L}\right)\vspace{5pt}\\
 -   \dfrac{  a_{uy}  \pi u_y}{L} \sin\left( \dfrac{ a_{uy}  \pi  y}{L}\right)
\end{array} \right],
\end{equation*}
\begin{equation*}
\nabla  \tv= \left[ \begin{array}{c}
-  \dfrac{  a_{vx}  \pi v_x}{L}  \sin\left( \dfrac{ a_{vx}  \pi  x}{L}\right)\vspace{5pt}\\
    \dfrac{  a_{vy}  \pi v_y}{L} \cos\left( \dfrac{ a_{vy}  \pi  y}{L}\right)
\end{array} \right]
\quad\mbox{and}\quad
\nabla \sa = \left[ \begin{array}{c}
-\dfrac{  a_{\sa x}  \pi  \nu_{\tsa x} }{L} \sin\left(\frac{a_{\sa x} \pi x}{L}\right)\vspace{5pt}\\
-  \dfrac{  a_{\sa y}  \pi  \nu_{\tsa x}}{L}  \sin\left(\frac{a_{\sa y} \pi y}{L}\right)
\end{array} \right].
\end{equation*}

\subsection{C Files}Files containing $C$ codes for the source terms have also been automatically generated. They are:\\ \texttt{FANS\_SA\_transient\_2d\_rho\_code.C, FANS\_SA\_transient\_2d\_u\_code.C, FANS\_SA\_transient\_2d\_v\_code.C,\\ FANS\_SA\_transient\_2d\_E\_code.C} and \texttt{FANS\_SA\_transient\_2d\_nu\_code.C}.

%\newpage
An example of the $C$ file from the source term for the 2D total energy source term $Q_{\tE}$~is:

\begin{footnotesize}
\begin{verbatim}
#include <math.h>

double SourceQ_e (double x, double y, double t, double mu, double c_v1, double cp, double cv, double Pr_t, double Pr)
{
  double Q_E;
  double RHO;
  double U;
  double V;
  double P;
  double NU_SA;
  double chi;
  double f_v1;
  double R;
  double mu_t;

  NU_SA = nu_sa_0 + nu_sa_x * cos(a_nusax * PI * x / L) + nu_sa_y * cos(a_nusay * PI * y / L)
    + nu_sa_t * cos(a_nusat * PI * t / L);
  RHO = r * rho_0 + rho_x * sin(a_rhox * PI * x / L) + rho_y * cos(a_rhoy * PI * y / L)
    + rho_t * sin(a_rhot * PI * t / L);
  U = u_0 + u_x * sin(a_ux * PI * x / L) + u_y * cos(a_uy * PI * y / L) + u_t * cos(a_ut * PI * t / L);
  V = v_0 + v_x * cos(a_vx * PI * x / L) + v_y * sin(a_vy * PI * y / L) + v_t * sin(a_vt * PI * t / L);
  P = p_0 + p_x * cos(a_px * PI * x / L) + p_y * sin(a_py * PI * y / L) + p_t * cos(a_pt * PI * t / L);
  chi = RHO * NU_SA / mu;
  f_v1 = pow(chi, 0.3e1) / (pow(chi, 0.3e1) + pow(c_v1, 0.3e1));
  mu_t = RHO * NU_SA * f_v1;
  R = cp - cv;

  Q_E = -a_ut * PI * u_t * RHO * U * sin(a_ut * PI * t / L) / L
  + a_vt * PI * v_t * RHO * V * cos(a_vt * PI * t / L) / L
  + (U * U + V * V) * a_rhox * PI * rho_x * U * cos(a_rhox * PI * x / L) / L / 0.2e1
  - (U * U + V * V) * a_rhoy * PI * rho_y * V * sin(a_rhoy * PI * y / L) / L / 0.2e1
  + (0.4e1 / 0.3e1 * a_ux * a_nusax * u_x * nu_sa_x * cos(a_ux * PI * x / L) * sin(a_nusax * PI * x / L)
    - a_uy * a_nusay * u_y * nu_sa_y * sin(a_uy * PI * y / L) * sin(a_nusay * PI * y / L)
    - a_vx * a_nusay * v_x * nu_sa_y * sin(a_vx * PI * x / L) * sin(a_nusay * PI * y / L)
    - 0.2e1 / 0.3e1 * a_vy * a_nusax * v_y * nu_sa_x * cos(a_vy * PI * y / L) * sin(a_nusax * PI * x / L))
    * PI * PI * f_v1 * RHO * U * pow(L, -0.2e1)
  + (-0.2e1 / 0.3e1 * a_ux * a_nusay * u_x * nu_sa_y * cos(a_ux * PI * x / L) * sin(a_nusay * PI * y / L)
    - a_uy * a_nusax * u_y * nu_sa_x * sin(a_uy * PI * y / L) * sin(a_nusax * PI * x / L)
    - a_vx * a_nusax * v_x * nu_sa_x * sin(a_vx * PI * x / L) * sin(a_nusax * PI * x / L)
    + 0.4e1 / 0.3e1 * a_vy * a_nusay * v_y * nu_sa_y * cos(a_vy * PI * y / L) * sin(a_nusay * PI * y / L))
    * PI * PI * f_v1 * RHO * V * pow(L, -0.2e1)
  + (-0.4e1 / 0.3e1 * a_rhox * a_ux * rho_x * u_x * cos(a_rhox * PI * x / L) * cos(a_ux * PI * x / L)
    + 0.2e1 / 0.3e1 * a_rhox * a_vy * rho_x * v_y * cos(a_rhox * PI * x / L) * cos(a_vy * PI * y / L)
    - a_rhoy * a_uy * rho_y * u_y * sin(a_rhoy * PI * y / L) * sin(a_uy * PI * y / L)
    - a_rhoy * a_vx * rho_y * v_x * sin(a_rhoy * PI * y / L) * sin(a_vx * PI * x / L))
    * PI * PI * f_v1 * U * NU_SA * pow(L, -0.2e1)
  + (a_rhox * a_uy * rho_x * u_y * cos(a_rhox * PI * x / L) * sin(a_uy * PI * y / L)
    + a_rhox * a_vx * rho_x * v_x * cos(a_rhox * PI * x / L) * sin(a_vx * PI * x / L)
    - 0.2e1 / 0.3e1 * a_rhoy * a_ux * rho_y * u_x * sin(a_rhoy * PI * y / L) * cos(a_ux * PI * x / L)
    + 0.4e1 / 0.3e1 * a_rhoy * a_vy * rho_y * v_y * sin(a_rhoy * PI * y / L) * cos(a_vy * PI * y / L))
    * PI * PI * f_v1 * V * NU_SA * pow(L, -0.2e1)
  + (U * U + V * V) * a_rhot * PI * rho_t * cos(a_rhot * PI * t / L) / L / 0.2e1
  + (f_v1 + pow(c_v1, 0.3e1) / (pow(chi, 0.3e1) + pow(c_v1, 0.3e1)))
    * (0.4e1 * a_ux * a_ux * u_x * sin(a_ux * PI * x / L) + 0.3e1 * a_uy * a_uy * u_y * cos(a_uy * PI * y / L))
    * PI * PI * mu * U * pow(L, -0.2e1) / 0.3e1
  + (f_v1 + pow(c_v1, 0.3e1) / (pow(chi, 0.3e1) + pow(c_v1, 0.3e1)))
    * (0.3e1 * a_vx * a_vx * v_x * cos(a_vx * PI * x / L) + 0.4e1 * a_vy * a_vy * v_y * sin(a_vy * PI * y / L))
    * PI * PI * mu * V * pow(L, -0.2e1) / 0.3e1
  - (a_uy * u_y * sin(a_uy * PI * y / L) + a_vx * v_x * sin(a_vx * PI * x / L)) * PI * RHO * U * V / L
  + (a_ux * u_x * cos(a_ux * PI * x / L) + a_vy * v_y * cos(a_vy * PI * y / L)) * cp * PI * P / L / R
  - (0.4e1 * a_ux * a_ux * u_x * u_x * pow(cos(a_ux * PI * x / L), 0.2e1)
    - 0.4e1 * a_ux * a_vy * u_x * v_y * cos(a_ux * PI * x / L) * cos(a_vy * PI * y / L)
    + 0.3e1 * a_uy * a_uy * u_y * u_y * pow(sin(a_uy * PI * y / L), 0.2e1)
    + 0.6e1 * a_uy * a_vx * u_y * v_x * sin(a_uy * PI * y / L) * sin(a_vx * PI * x / L)
    + 0.3e1 * a_vx * a_vx * v_x * v_x * pow(sin(a_vx * PI * x / L), 0.2e1)
    + 0.4e1 * a_vy * a_vy * v_y * v_y * pow(cos(a_vy * PI * y / L), 0.2e1))
    * PI * PI * mu_t * pow(L, -0.2e1) / 0.3e1
  + (0.4e1 * a_ux * a_ux * u_x * sin(a_ux * PI * x / L) + 0.3e1 * a_uy * a_uy * u_y * cos(a_uy * PI * y / L))
    * PI * PI * mu_t * U * pow(L, -0.2e1) / 0.3e1
  + (0.3e1 * a_vx * a_vx * v_x * cos(a_vx * PI * x / L) + 0.4e1 * a_vy * a_vy * v_y * sin(a_vy * PI * y / L))
    * PI * PI * mu_t * V * pow(L, -0.2e1) / 0.3e1
  + (0.3e1 * a_ux * u_x * cos(a_ux * PI * x / L) + a_vy * v_y * cos(a_vy * PI * y / L)) * PI * RHO * U * U / L / 0.2e1
  + (a_ux * u_x * cos(a_ux * PI * x / L) + 0.3e1 * a_vy * v_y * cos(a_vy * PI * y / L)) * PI * RHO * V * V / L / 0.2e1
  - (f_v1 + pow(c_v1, 0.3e1) / (pow(chi, 0.3e1) + pow(c_v1, 0.3e1)))
    * (0.4e1 * a_ux * a_ux * u_x * u_x * pow(cos(a_ux * PI * x / L), 0.2e1)
    - 0.4e1 * a_ux * a_vy * u_x * v_y * cos(a_ux * PI * x / L) * cos(a_vy * PI * y / L)
    + 0.3e1 * a_uy * a_uy * u_y * u_y * pow(sin(a_uy * PI * y / L), 0.2e1)
    + 0.6e1 * a_uy * a_vx * u_y * v_x * sin(a_uy * PI * y / L) * sin(a_vx * PI * x / L)
    + 0.3e1 * a_vx * a_vx * v_x * v_x * pow(sin(a_vx * PI * x / L), 0.2e1)
    + 0.4e1 * a_vy * a_vy * v_y * v_y * pow(cos(a_vy * PI * y / L), 0.2e1)) * PI * PI * mu * pow(L, -0.2e1) / 0.3e1
  - (mu_t / Pr_t + mu / Pr) * (-(a_px * a_px * p_x * cos(a_px * PI * x / L)
    + a_py * a_py * p_y * sin(a_py * PI * y / L)) * cp * PI * PI * pow(L, -0.2e1) / R / RHO
  + (a_rhox * a_rhox * rho_x * sin(a_rhox * PI * x / L) + a_rhoy * a_rhoy * rho_y * cos(a_rhoy * PI * y / L))
    * cp * PI * PI * P * pow(L, -0.2e1) / R * pow(RHO, -0.2e1))
  + cv * a_rhot * PI * rho_t * P * cos(a_rhot * PI * t / L) / L / R / RHO
  - (a_px * a_nusax * p_x * nu_sa_x * sin(a_px * PI * x / L) * sin(a_nusax * PI * x / L)
    - a_py * a_nusay * p_y * nu_sa_y * cos(a_py * PI * y / L) * sin(a_nusay * PI * y / L))
    * cp * PI * PI * mu_t * pow(L, -0.2e1) / Pr_t / R / RHO / NU_SA
  - (a_rhox * a_nusax * rho_x * nu_sa_x * cos(a_rhox * PI * x / L) * sin(a_nusax * PI * x / L)
    - a_rhoy * a_nusay * rho_y * nu_sa_y * sin(a_rhoy * PI * y / L) * sin(a_nusay * PI * y / L))
    * cp * PI * PI * mu_t * P * pow(L, -0.2e1) / Pr_t / R * pow(RHO, -0.2e1) / NU_SA
  - (a_rhox * a_px * rho_x * p_x * cos(a_rhox * PI * x / L) * sin(a_px * PI * x / L)
    + a_rhoy * a_py * rho_y * p_y * sin(a_rhoy * PI * y / L) * cos(a_py * PI * y / L))
    * (Pr * mu_t + 0.2e1 * Pr_t * mu) * cp * PI * PI / Pr / Pr_t * pow(L, -0.2e1) / R * pow(RHO, -0.2e1)
  - (a_rhox * a_rhox * rho_x * rho_x * pow(cos(a_rhox * PI * x / L), 0.2e1)
    + a_rhoy * a_rhoy * rho_y * rho_y * pow(sin(a_rhoy * PI * y / L), 0.2e1))
    * (Pr * mu_t + 0.2e1 * Pr_t * mu) * cp * PI * PI * P / Pr / Pr_t * pow(L, -0.2e1) / R * pow(RHO, -0.3e1)
  + cp * a_py * PI * p_y * V * cos(a_py * PI * y / L) / L / R
  - cp * a_px * PI * p_x * U * sin(a_px * PI * x / L) / L / R;
}
\end{verbatim}

\end{footnotesize}

%---------------------------------------------------------------------------------------------------------
\bibliographystyle{chicago} 
\bibliography{../../MMS_bib}

\end{document}

